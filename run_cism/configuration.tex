\section{Model Configuration}
\label{ug.sec.config}

General model configuration options 
specify the grid and time-stepping used by the model, 
the dynamical core (dycore) used, 
and control various optional physics packages and parameter values.
The \texttt{[grid]} and \texttt{[time]} configuration sections are always required.
Also, while not required, in almost all situations  \texttt{[options]} 
and (if using a higher-order dycore) \texttt{[ho\_options]} will be included in .config files.
The \texttt{[parameters]} and \texttt{[sigma]} sections are also commonly used.  
The \texttt{[GTHF]}, \texttt{[isostasy]}, and \texttt{[projection]} sections
are needed only if the associated features are desired.  Details of each of these
sections, what they control, and the available options for each section are listed 
in the tables below.

\begin{center}
  \tablefirsthead{%
    \hline
  }
  \tablehead{%
    \hline
    \multicolumn{2}{|l|}{\emph{\small continued from previous page}}\\
    \hline
  }
  \tabletail{%
    \hline
    \multicolumn{2}{|r|}{\emph{\small continued on next page}}\\
    \hline}
  \tablelasttail{\hline}
  \begin{supertabular*}{\textwidth}{@{\extracolsep{\fill}}|l|p{10cm}|}


%%%% GRID
    \hline
    \multicolumn{2}{|l|}{\texttt{[grid]}}\\
    \hline
    \multicolumn{2}{|p{0.97\textwidth}|}{Define model grid. 
%Maybe we should make this optional and read grid specifications from input netCDF file (if present). Certainly, the input netCDF files should be checked (but presently are not) if grid specifications are compatible.
}\\
    \hline
    \texttt{ewn} & (integer) number of nodes in $x$--direction\\
    \texttt{nsn} & (integer) number of nodes in $y$--direction\\
    \texttt{upn} & (integer) number of nodes in $z$--direction\\
    \texttt{dew} & (real) node spacing in $x$--direction (m)\\
    \texttt{dns} & (real) node spacing in $y$--direction (m)\\
%% global_bc
    \texttt{global\_bc} & 
        boundary conditions for the edges of the global domain \\ &
    \begin{tabular}[t]{cp{0.85\linewidth}}
      {\bf 0} & periodic \\
      1 & outflow \\
    \end{tabular}\\
%    \texttt{sigma\_file} & (string) Name of file containing $\sigma$ coordinates. Alternatively, the sigma levels may be specified using the \texttt{[sigma]} section decribed below. If no sigma coordinates are specified explicitly, they are calculated based on the value of \texttt{sigma\_builtin} \\
    \texttt{sigma} &
%    \begin{tabular}[t]{cp{\linewidth}}
%      \multicolumn{2}{p{0.72\textwidth}}{If sigma coordinates are not specified in this configuration file or using the \texttt{sigma\_file} option, this specifies how to compute the sigma coordinates.} \\
      method for specifying sigma coordinates:  \\ &
    \begin{tabular}[t]{cl}
      {\bf 0} & Use Glimmer's default spacing \\[0.05in] 
        & $\sigma_i=\frac{1-(x_i+1)^{-n}}{1-2^{-n}}\quad\mbox{with}\quad x_i=\frac{\sigma_i-1}{\sigma_n-1}, n=2.$ \\[0.05in]
      1 & use sigma coordinates defined in external file (named sigma\_file) \\
      2 & use sigma coordinates given in configuration file \\
      3 & use evenly spaced sigma levels (required by the Glam dycore) \\
      4 & use Pattyn sigma levels \\
    \end{tabular}\\



%%%% SIGMA
    \hline
    \hline
    \hline
    \multicolumn{2}{|l|}{\texttt{[sigma]}}\\
    \hline
    \multicolumn{2}{|p{0.95\textwidth}|}{Define the sigma levels used in the vertical discretization (\texttt{sigma}=2 above). This is an alternative to using a separate file (specified in section \texttt{[grid]} above). If neither is used, the levels are calculated as described above.} \\
    \hline
    \texttt{sigma\_levels} & (real) list of sigma levels, in ascending order, separated by spaces. These run between 0.0 and 1.0. \\



%%%% TIME
    \hline
    \hline
    \hline
    \multicolumn{2}{|l|}{\texttt{[time]}}\\
    \hline
    \multicolumn{2}{|p{0.95\textwidth}|}{Configure time steps and diagnostic specifications} \\
    %{\bf TODO: address / remove this next note:} Update intervals should probably become absolute values rather than related to the main time step when we introduce variable time steps. \textbf{Steve: I don't actually know what the question is here or what this note is supposed to be about.}}\\
    \hline
    \texttt{tstart} & (real) start time of the model in years\\
    \texttt{tend} & (real) end time of the model in years\\
    \texttt{dt} & (real) size of time step in years\\
    \texttt{subcyc} & (integer) number of time steps to subcycle evolution within dt using a steady velocity field \\
    \texttt{ntem} & (real) time step multiplier setting the ice temperature update interval\\
%    \texttt{nvel} & (real) time step multiplier setting the velocity update interval\\
%    \texttt{profile} & (integer) profile frequency (number of time steps) {\bf TODO: more useful description of what profiling is? Steve: I've never used this either.   
%    Matt, my guess would be that Pat's profiling code is superior to this and, if we document that anywhere (do we?), we should support using that instead.}\\
%   Matt: Disabling this for 2.0 release
    \texttt{dt\_diag} & (real) writing diagnostic variables to log file every dt\_diag yrs\\
    \texttt{idiag} & (int) $x$ direction index for diagnostic grid point in log file\\
    \texttt{jdiag} & (int) $y$ direction index for diagnostic grid point in log file\\
%    \texttt{ndiag} & (int) number of time steps between diagnostics. {\bf DEPRECATED}.  Use \texttt{dt\_diag}. \textbf{Steve: Should we then remove and replace this with dt\_diag rather than using the old var name?}\\



%%%% Options
    \hline
    \hline
    \hline
    \multicolumn{2}{|l|}{\texttt{[options]}}\\
    \hline
    \multicolumn{2}{|p{0.95\textwidth}|}{Parameters set in this section determine how various components of the ice sheet model are treated.  Configuration number options with a $\dagger$ are specific to the higher-order dycores (e.g., Glissade). Options with a ? are working, but are currently not scientifically supported, and are therefore for use at your own risk. Options with a ! are in development and will be supported in future code releases. }\\
    \hline
%% dycore
    \texttt{dycore} & 
    \begin{tabular}[t]{lp{0.85\linewidth}}
      0 & Glide (1-processor, 3d, shallow-ice-approximation dycore) \\
      1$\dagger$? & Glam (parallel, 3d, FDM, 1st-order-accurate dycore)  \\
      {\bf 2$\dagger$} & Glissade (parallel, 3d, FEM, 1st-order-accurate dycore)  \\
      3$\dagger$! & FELIX (parallel, 3d, FEM, 1st-order-accurate dycore)  \\
      4$\dagger$! & BISICLES (parallel, quasi-3d, FVM, L1L2 dycore)  \\
    \end{tabular}\\
%% evolution
    \texttt{evolution} (ice thickness) & 
    \begin{tabular}[t]{lp{0.85\linewidth}}
      {\bf 0} & pseudo-diffusion (Glide only)\\
      1 & ADI scheme  (Glide only)\\
      2 & diffusion (Glide only)\\
      3$\dagger$ & incremental remapping \\
      4$\dagger$ & first-order upwind  \\
      5$\dagger$ & evolve without changing ice thickness (Useful for running with a fixed geometry, e.g. for a temperature spinup. On each time step, geometry and tracers are evolved using incremental remapping, after which geometry is reset to its initial value. This evolution scheme is still subject to the advective CFL condition.)\\
    \end{tabular}\\
%% temperature
    \texttt{temperature} & 
    \begin{tabular}[t]{lp{0.85\linewidth}}
      0 & Set each ice column to local surface air temperature \\
      {\bf 1} & prognostic temperature calculation \\
      2 & hold temperature steady at initial value \\
      3! & prognostic temperature calculation using enthalpy-based formulation \\
    \end{tabular}\\
%% temperature init
    \texttt{temp\_init} & 
    \begin{tabular}[t]{lp{0.85\linewidth}}
      0 & initial temperatures isothermal at $0^\circ$C\\
      {\bf 1} & initial column temperatures set to atmos. temperature \\
      2 & initial column temperatures linearly interpolated between atmos. temperature and pressure melting point\\
    \end{tabular}\\
%% flow law
    \texttt{flow\_law} &  
    \begin{tabular}[t]{lp{0.85\linewidth}}
      {\bf 0}  & constant (using the value of \texttt{default\_flwa)}\\
      1 & temperature-dependent, \citet{PatersonBudd:1982} ($T=-5^\circ$C)\\
      2 & temperature-dependent, \citet{PatersonBudd:1982} (function of variable T)\\
    \end{tabular}\\
%% basal water
    \texttt{basal\_water} & 
    \begin{tabular}[t]{lp{0.85\linewidth}}
      {\bf 0} & none \\
      1 & local water balance\\
      2? & compute the steady-state, routing-based, basal water flux and water layer thickness (NOTE: not supported for $> 1$ processor) \\
      3 & use a constant basal water layer thickness everywhere, to enforce T=T${_{pmp}}$ everywhere \\
      4! & ocean penetration parameterization for effective pressure from \citet{Leguy2014} \\
    \end{tabular}\\
%% basal mass balance
    \texttt{basal\_mass\_balance} & 
    \begin{tabular}[t]{lp{0.85\linewidth}}
      {\bf 0} & ignore basal melt rate in mass balance calculation \\
      1 & include basal melt rate in mass balance calculation \\
    \end{tabular}\\
%% slip coefficient
    \texttt{slip\_coeff} & 
        slip coefficient (Glissade local SIA and Glide \textit{only}) \\ &
    \begin{tabular}[t]{lp{0.85\linewidth}}
      {\bf 0} & zero (no sliding) \\
      1 & set to a non--zero constant everywhere\\
      2 & set to constant where basal water (bwat) is nonzero\\
      3 & set to constant where the ice base is melting\\
      4 & set proportional to basal melt rate\\
      5 & \texttt{tanh} function of basal water (bwat)\\
    \end{tabular}\\
%% marine margin
    \texttt{marine\_margin} & 
    \begin{tabular}[t]{lp{0.85\linewidth}}
      0 & ignore marine margin\\
      {\bf 1} & set thickness to zero if floating\\
      2 & lose fraction of ice from edge cells\\
      3 & set thickness to zero if relaxed bedrock is below a given depth (variable ``mlimit" in glide\_types)\\
      4 & set thickness to zero if present-day bedrock is below a given depth (variable ``mlimit" in glide\_types)\\
      5? & Huybrechts calving scheme \\
    \end{tabular}\\
%% vertical integration
    \texttt{vertical\_integration} & 
       (Glide \textit{only}) \\ &
    \begin{tabular}[t]{lp{0.85\linewidth}}
      {\bf 0} & standard integration (to obtain vertical velocity profile)\\
      1 & constrained to obey kinematic velocity at upper surface boundary\\
    \end{tabular}\\
%% geothermal heat flux
    \texttt{gthf} &  
    \begin{tabular}[t]{lp{0.85\linewidth}}
      {\bf 0} & prescribed, uniform geothermal heat flux \\
      1 & read 2d geothermal heat flux field from input file \\
      2 & calculate geothermal heat flux using 3d diffusion model \\
    \end{tabular}\\
%% isostasy
    \texttt{isostasy} &  
    \begin{tabular}[t]{lp{0.85\linewidth}}
      {\bf 0} & no isostatic adjustment \\
      1 & compute isostatic adjustment using lithosphere / asthenosphere model (see below for available options)  \\
    \end{tabular}\\
%% topography
    \texttt{topo\_is\_relaxed} &  
    \begin{tabular}[t]{lp{0.85\linewidth}}
      {\bf 0} & relaxed topography is read from a separate input variable, \texttt{relx} \\
      1 & first time slice of input topography is assumed to be relaxed\\
      2 & first time slice of input topography is assumed to be in isostatic
      equilibrium with ice thickness \\
    \end{tabular}\\
%% periodicity					%% no longer used / supported
%    \texttt{periodic\_ew} & 
%  {\bf TODO: Is this still used?  If so, do we need to add perdiodic\_ns?} \\&
%    \begin{tabular}[t]{lp{0.85\linewidth}}
 %     {\bf 0} & switched off\\
 %     1 & periodic lateral EW boundary conditions (Glide  dycore \textit{only}) \\
 %   \end{tabular}\\
%% restart
    \texttt{restart} &
    \textit{Note:} alternate keyword {\bf hotstart} is retained for backwards compatibility. \\ &
    \begin{tabular}[t]{lp{0.85\linewidth}}
      {\bf 0} & normal start (initial values taken from input file or, if absent, using default options)\\
      1 & restart model using input from previous run;
           specific fields required for restart are dependent on chosen options (add ``restart" to the 
           \texttt{variable} list in the \texttt{[CF output]} section of the \texttt{.config} file to automatically save the appropriate restart fields.)\\
    \end{tabular}\\
    \hline
    \texttt{ioparams} & (string) name of file containing netCDF I/O configuration. The main configuration file is searched for I/O related sections if no file name is given (default).  In other words, you can remove sections \texttt{CF input}, \texttt{CF output}, and \texttt{CF forcing} from the primary configuration file and place them in a separate file, the path to which is specified here.\\



 %%%% HIGHER-ORDER OPTIONS
    \hline
    \hline
    \hline
    \multicolumn{2}{|l|}{\texttt{[ho\_options]}}\\
    \hline
    \multicolumn{2}{|p{0.95\textwidth}|}{Options set in this section determine how various components of the higher-order extensions to the ice sheet model (e.g., Glissade) are treated. Defaults are indicated in bold. These options have no effect on the shallow-ice (Glide) dycore. In this section, options with a ? are working but are currently not scientifically supported (and are therefore for use at your own risk). Options marked with a * apply only to a serial build (or a parallel build if run on 1 processor). Options marked with a ! are under development and will be supported in future versions of the code (hence, these are also for use at your own risk).}\\
    \hline
%% which_ho_nonlinear
    \texttt{which\_ho\_nonlinear} & 
    \begin{tabular}[t]{lp{0.85\linewidth}}
      {\bf 0} & treat nonlinearity in momentum balance using Picard iteration \\
      1? & treat nonlinearity in momentum balance using Jacobian-Free Newton-Krylov iteration (Glam only)  \\
    \end{tabular}\\     
%% which_ho_sparse
    \texttt{which\_ho\_sparse} & 
%\\ &
    \begin{tabular}[t]{lp{0.85\linewidth}}
      -1* & solve sparse linear system using SLAP with incomplete Cholesky preconditioned conjugate gradient method\\
      0* & solve sparse linear system using SLAP with incomplete LU-preconditioned biconjugate gradient method\\
      1* & solve sparse linear system using SLAP with incomplete LU-preconditioned GMRES method\\
      2 & solve sparse linear system using preconditioned conjugate gradient method: standard algorithm (Glissade only) \\
      {\bf 3} & solve sparse linear system using preconditioned conjugate gradient method: Chronopoulos-Gear algorithm (Glissade only)\\
      4 & solve sparse linear system using \textit{Trilinos}, incomplete LU-preconditioned GMRES method (\textit{Trilinos}-compatible build only)\\
    \end{tabular}\\     
%% which_ho_efvs
    \texttt{which\_ho\_efvs} & 
    \begin{tabular}[t]{lp{0.85\linewidth}}
      0 & use a constant value for the effective viscosity (i.e., linear viscosity). The default value is 2336041 Pa yr (as used by ISMIP-HOM Test F).\\
      1 & set the effective viscosity to a value based on the flow rate factor: efvs $= 0.5 * A^{-1/n}$\\
      {\bf 2} & use the effective strain rate to compute the effective viscosity (i.e., full nonlinear treatment) \\
    \end{tabular}\\  
%% which_no_disp
    \texttt{which\_ho\_disp} & 
    \begin{tabular}[t]{lp{0.85\linewidth}}
      -1 & no dissipation term included in temperature equation \\
      0 & calculate dissipation in temperature equation assuming SIA ice dynamics \\
      {\bf 1} & calculate dissipation in temperature equation assuming first-order ice dynamics \\
    \end{tabular}\\    
%% which_ho_babc
    \texttt{which\_ho\_babc} & 
        Implementation of basal boundary condition in higher-order dycore \\ &
    \begin{tabular}[t]{lp{0.85\linewidth}}
      0 & constant value of ``beta" \\
      1 & specify a simple pattern for ``beta" (hardcoded, mainly useful for debugging)\\
      2 & read map of yield stress (in Pa) from input field ``mintauf" to simulate sliding 
          over a plastic subglacial till (Picard-based solution) \\
      3 & calculate ``beta" as linear (inverse) function of basal water thickness\\
      {\bf 4} & (virtually) no slip everywhere in domain (``beta" set to very large value)\\
      5 & read map of ``beta" from .nc input file using standard I/O \\
      6 & no slip everywhere in domain (using Dirichlet basal BC)\\
      7! & read map of yield stress (in Pa) from input field ``mintauf" to simulate sliding 
          over a plastic subglacial till (Newton-based solution)\\
      8* & Spatial field of ``beta" required for ISMIP-HOM Test C (avoids interpolation error 
          associated with option 5; works for a single processor only) \\
      9! & Weertman-style power-law accounting for effective pressure (Eq. \ref{weertmansliding2}) \\
      10! & Coulomb friction law (Eq. \ref{CF-law}) \\
    \end{tabular}\\  
%% which_ho_resid
    \texttt{which\_ho\_resid} &
     Residual calculation method for higher-order velocity solvers (e.g., Glissade). 
     Nonlinear iterations are halted once the residual falls below a specified value. \\ &
    \begin{tabular}[t]{lp{0.85\linewidth}}
      0? & use the maximum value of the normalized velocity vector update, defined by 
      $r = \frac{|vel_{k-1} - vel_k|}{vel_k}$ \\
      1? & as in option 0 but omitting the basal velocities from the comparison
          (useful in cases where an approx. no slip basal BC is enforced) \\
      2? & as in option 0 but using the mean rather than the max \\
      {\bf 3} & use the L2 norm of the system residual, defined by $r = Ax - b$ \\
      4  & use L2 norm of residual relative to rhs, $|Ax - b|/|b|$
    \end{tabular}\\  
%% which_ho_approx
    \texttt{which\_ho\_approx} &
     Stokes-flow approximation to use with Glissade dycore \\ &
    \begin{tabular}[t]{lp{0.85\linewidth}}
      -1 & local shallow-ice approximation, Glide-type calculation (uses glissade\_velo\_sia) \\
      0 & 3d matrix shallow-ice approximation, vertical-shear stresses only (uses glissade\_velo\_higher) \\
      1 & shallow-shelf approximation (SSA) with horizontal-plane stresses only (uses glissade\_velo\_higher; requires \texttt{which\_ho\_precond} $<=$1) \\
      {\bf 2} & Blatter-Pattyn with both vertical-shear and horizontal-plane stresses (uses glissade\_velo\_higher) \\
       3 & depth-integrated (L1L2) approximation, with both vertical shear and horizontal-plane stresses (uses glissade\_velo\_higher; requires \texttt{which\_ho\_precond} $<=$1) \\
    \end{tabular}\\  
%% which_ho_precond
    \texttt{which\_ho\_precond} &
     Preconditioner to use in the linear PCG solve of the Glissade dycore \\ &
    \begin{tabular}[t]{lp{0.85\linewidth}}
      0 & no preconditioner \\
      1 & diagonal preconditioner \\
      {\bf 2} & physics-based (shallow-ice) preconditioner (not valid for SSA and L1L2) \\
    \end{tabular}\\  
%% which_ho_gradient
    \texttt{which\_ho\_gradient} &
     Which spatial gradient operator to use in the Glissade dycore \\ &
    \begin{tabular}[t]{lp{0.85\linewidth}}
      {\bf 0} & centered gradient \\
      1 & upstream gradient (damps checkerboard noise in prognostic simulations) \\
    \end{tabular}\\  
%% which_ho_gradient_margin
    \texttt{which\_ho\_gradient\_margin} &
     Spatial gradient operator to use in the Glissade dycore at ice sheet margins. \\ &
    \begin{tabular}[t]{lp{0.85\linewidth}}
      0 & use information from all neighboring cells, ice-covered or ice-free \\
      {\bf 1} & use information from ice-covered and/or land cells, but not ice-free ocean cells \\
      2 & use information from ice-covered cells only \\
    \end{tabular}\\  
%% which_ho_assemble_beta
    \texttt{which\_ho\_assemble\_beta} &
     Finite-element assembly method for basal boundary conditions that use ``beta" field \\ &
    \begin{tabular}[t]{lp{0.85\linewidth}}
      {\bf 0} & Standard finite-element calculation, which effectively applies a smoothing to ``beta" (and ``mintauf") \\
      1 &  Apply the local ``beta" (or ``mintauf") value at each vertex (no smoothing) \\ 
    \end{tabular}\\  
%% glissade_maxiter
    \texttt{glissade\_maxiter} &
    \begin{tabular}[t]{lp{0.85\linewidth}}
	{\bf 100} & Maximum number of nonlinear (Picard) iterations in the Glissade dycore \\ 
    \end{tabular}\\  



% MJH: Commenting out external dycore options for 2.0 release.
%% %%%% EXTERNAL DYCORE OPTIONS
%    \hline
%    \hline
%    \hline
%    \multicolumn{2}{|l|}{\texttt{[external\_dycore\_options]}}\\
%    \hline
%    \multicolumn{2}{|p{0.95\textwidth}|}{Options set in this section are specific to external dycores that 
%    may have additional options/option files.  {\bf TODO: include this stuff?  elaborate?} }\\
%    \hline
%%% external_dycore_type
%    \texttt{external\_dycore\_type} & 
%    \begin{tabular}[t]{lp{0.85\linewidth}}
%      1 & Use BISICLES external dycore \\
%      2 & Use Albany-FELIX external dycore \\
%    \end{tabular}\\     
%    \texttt{dycore\_input\_file} &
%    Specify path to additional configuration file required by the external dycore. \\



%%%% PARAMETERS
    \hline
    \hline
    \hline
    \multicolumn{2}{|l|}{\texttt{[parameters]}}\\
    \hline
    \multicolumn{2}{|p{0.95\textwidth}|}{Set values for various parameters.  Parameters with a $\dagger$ are specific to the higher-order dycores (e.g., Glissade).}\\
    \hline
    \texttt{log\_level} & (integer) set to a value between 0, no messages, and 6, all messages are displayed to stdout. By default, messages are only logged to a file.
    The format for this filename is ``configuration-file-name.config.log" \\
    \texttt{ice\_limit} & (real) below this limit ice is only accumulated/ablated; ice dynamics are switched on once the ice thickness is above this value. (default = 100.0 m) \\
    \texttt{ice\_limit\_temp} $\dagger$ & (real) minimum thickness for computing vertical temperature (m). (default = 1.0 m) \\
    \texttt{marine\_limit} & (real) all ice is assumed lost (calved) once water depths reach this value (for \texttt{marine\_margin}=3 or 4 in 
    \texttt{[options]} above). Note, water depth is negative.  (default = -200.0 m) \\
    \texttt{calving\_fraction} & (real) fraction of ice lost due to calving (for \texttt{marine\_margin}=2). (default = 0.8)\\
    \texttt{geothermal} & (real) constant geothermal heat flux, positive down by convention (hence $<$ 0). (default = -0.05 W m$^{-2}$)\\
    \texttt{flow\_factor} & (real) the flow law rate factor is multiplied by this factor (default = 1.0; in previous versions of Glimmer-CISM the default value was 3.0) \\
    \texttt{default\_flwa} & flow law parameter A to use in isothermal experiments (flow\_law set to 0).  Default value is $10^{-16}$ Pa$^{-n}$ yr$^{-1}$. This 
    overrides any temperature dependence. \\
    \texttt{efvs\_constant} $\dagger$ & Constant value of effective viscosity when using \texttt{which\_ho\_efvs}=0. Default value is 2336041 Pa yr, as in ISMIP-HOM Test F. \\
%    \texttt{hydro\_time} & (real) basal hydrology time constant (default = 1000.0 yr) {\bf TODO: IS THIS STILL USED? Steve: I grepped for it in a few different dirs. The only place
%    I saw it was in glide setup, suggesting that it is not being used right now.}\\
    \texttt{basal\_tract\_const} & constant basal traction parameter. You can load a .nc file with a variable called \texttt{soft} if you want a spatially varying 
    bed softness parameter (Glissade local SIA and Glide only) \\
    \texttt{basal\_tract\_max} & max value for basal traction when using \texttt{slip\_coeff}=4. \\
    \texttt{basal\_tract\_slope} & slope value for basal traction relation when using \texttt{slip\_coeff}=4. (Relation also uses \texttt{basal\_tract\_const}.)\\
    \texttt{basal\_tract\_tanh} & (real(5)) basal traction factors. Basal traction is set to $B=\tanh(W)$ with the parameters
      \begin{tabular}{cp{\linewidth}}
       (1) & width of the $\tanh$ curve\\
       (2) & $W$ at midpoint of $\tanh$ curve [m]\\
       (3) & $B$ minimum [ma$^{-1}$Pa$^{-1}$] \\
       (4) & $B$ maximum [ma$^{-1}$Pa$^{-1}$] \\
       (5) & multiplier for marine sediments \\
     \end{tabular}\\
    \texttt{ho\_beta\_const} $\dagger$ & (real) spatially uniform beta used when \texttt{which\_ho\_babc} = 0. (default = 10.0 Pa yr m$^{-1}$) \\
    \texttt{friction\_powerlaw\_k} $\dagger$ & (real) friction coefficient $k$ used for \texttt{which\_ho\_babc} = 9 (Eq. \ref{weertmansliding2}) (default = 8.4e-9 m y$^{-1}$ Pa$^{-2}$, from \citet{Bindschadler1983} converted to CISM units) \\
    \texttt{coulomb\_c} $\dagger$ & (real) Coulomb friction coefficient (unitless), $C$,
used for \texttt{which\_ho\_babc} = 10 (Eq. \ref{CF-law}) (default = 0.42, from \citet{Pimentel2010a}) \\
    \texttt{coulomb\_bump\_wavelength} $\dagger$ & (real) wavelength (m) of the dominant bedrock bumps, $\lambda_{max}$,
used for \texttt{which\_ho\_babc} = 10 (Eq. \ref{CF-law}) (default = 2.0 m, from \citet{Pimentel2010a}) \\
    \texttt{coulomb\_bump\_slope} $\dagger$ & (real) maximum slope (unitless) of the dominant bedrock bumps, $m_{max}$,
used for \texttt{which\_ho\_babc} = 10 (Eq. \ref{CF-law}) (default = 0.5 m, from \citet{Pimentel2010a}) \\
    \texttt{p\_ocean\_penetration} & (real) $p$-exponent in ocean penetration parameterization for (\texttt{basal\_water} = 4 (default = 0.0)\\
    \texttt{periodic\_offset\_ew}$\dagger$ & (real) vertical offset between east and west edges of the global domain. (default = 0.0 m)  (Primarily used for ISMIP-HOM and Stream test cases.) \\
    \texttt{periodic\_offset\_ns}$\dagger$ & (real) vertical offset between north and south edges of the global domain. (default = 0.0 m)  (Primarily used for ISMIP-HOM and Stream test cases.)\\



%%%% GTHF
    \hline
    \hline
    \hline
    \multicolumn{2}{|l|}{\texttt{[GTHF]}}\\
    \hline
    \multicolumn{2}{|p{0.95\textwidth}|}{Options related to lithospheric temperature and geothermal heat calculation.  Ignored unless \texttt{gthf} = 1.}\\
    \hline
    \texttt{num\_dim} & can be either \texttt{1} for 1D calculations or 3 for 3D calculations.\\
    \texttt{nlayer} & number of vertical layers (default: 20). \\
    \texttt{surft} & initial surface temperature (default 2$^\circ$C).\\
    \texttt{rock\_base} & depth below sea-level at which geothermal heat gradient is applied (default: -5000m).\\
    \texttt{numt} & number time steps for spinning up GTHF calculations (default: 0).\\
    \texttt{rho} & The density of lithosphere (default: 3300kg m$^{-3}$).\\
    \texttt{shc} & specific heat capcity of lithosphere (default: 1000J kg$^{-1}$ K$^{-1}$).\\
    \texttt{con} & thermal conductivity of lithosphere (3.3 W m$^{-1}$ K$^{-1}$).\\    



%%%% ISOSTASY
    \hline
    \hline
    \hline
    \multicolumn{2}{|l|}{\texttt{[isostasy]}}\\
    \hline
    \multicolumn{2}{|p{0.95\textwidth}|}{Options related to isostasy model. Ignored unless \texttt{isostasy} = 1. Options marked with a * work only with a serial build (or a parallel build if run on 1 processor).} \\
%    Steve: I tested these out and compared w/ older versions of the code (this was an old to-do item). As far as I know, all options ``work", but only the combinations using the local lithosphere (no flexural rigidity) work in parallel.
    \hline
    \texttt{lithosphere} & \begin{tabular}[t]{lp{0.9\linewidth}} 
      {\bf 0} & local lithosphere, equilibrium bedrock depression is found using Archimedes' principle \\
      1* & elastic lithosphere, flexural rigidity is taken into account
    \end{tabular} \\
    \texttt{asthenosphere} & \begin{tabular}[t]{lp{\linewidth}}
      {\bf 0} & fluid mantle, isostatic adjustment happens instantaneously \\
      1 & relaxing mantle, mantle is approximated by a half-space \\
    \end{tabular} \\    
    \texttt{relaxed\_tau} & characteristic time constant of relaxing mantle (default: 4000.a) \\
    \texttt{update} & lithosphere update period (default: 500.a) \\
    \hline
%%%%
%    \hline
%    \multicolumn{2}{|l|}{\texttt{[elastic lithosphere]}}\\
%    \hline
%    \multicolumn{2}{|p{0.95\textwidth}|}{Set up parameters of the elastic lithosphere.}\\
    \hline
    \texttt{flexural\_rigidity} & flexural rigidity of the lithosphere (default: 0.24e25 Pa m$^3$)\\



%%%% PROJECTION
    \hline
    \hline
    \hline
    \multicolumn{2}{|l|}{\texttt{[projection]}}\\
    \hline
    \multicolumn{2}{|p{0.95\textwidth}|}{Specify map projection for reference. The reader is
    referred to Snyder J.P. (1987) \emph{Map Projections - a working manual.} USGS 
        Professional Paper 1395.} \\
% TODO: I am assuming this works.  Code is not in glide\_setup.F90.  
% Appears to be in glimmap\_printproj() in ../libglimmer/glimmer\_map\_init.F90. 
    \hline
    \texttt{type} & string that specifies the projection type
    (\texttt{LAEA}, \texttt{AEA}, \texttt{LCC} or \texttt{STERE}). \\
    \texttt{centre\_longitude} & central longitude in degrees east \\
    \texttt{centre\_latitude} & central latitude in degrees north \\
    \texttt{false\_easting} & false easting in meters \\
    \texttt{false\_northing} & false northing in meters \\
    \texttt{standard\_parallel} & location of standard parallel(s) in degrees
    north. Up to two standard parallels may be specified (depending on the
    projection). \\
    \texttt{scale\_factor} & non-dimensional; relevant only for the stereographic projection \\

  \end{supertabular*}
\end{center}


