\label{ch:runcism}

\section{Overview of Running CISM}

Assuming you successfully completed the Installation instructions in Chapter \ref{chp:install},
the executable for running the model, \texttt{cism\_driver} can be found in your 
build directory in a subdirectory called \texttt{cism\_driver} 
(e.g., \texttt{./builds/mac-gnu/cism\_driver/cism\_driver}).

The build system creates the executable at this path but does not automatically
make it available to other locations on your system.  How you choose to do so depends 
on your situation.  See the introduction to Chapter \ref{sec:testcases} for 
an overview of how to make the executable available to other locations on your system
(e.g., symlinking, copying, modifying your PATH environment variable).

Unlike previous versions of Glimmer, with CISM 2.0 this single \texttt{cism\_driver} 
executable is used for running the model in all configurations.  \
texttt{cism\_driver} can be invoked with a single argument specifying 
a CISM .config file to run CISM  as a standalone ice sheet model without GLINT climate forcing,
 or with two arguments (a CISM config file and a GLINT config file) 
to run CISM with GLINT climate forcing:
\begin{verbatim}
 Call cism_driver with either 1 or 2 arguments. Examples:
 cism_driver ice_sheet.config
 cism_driver ice_sheet.config climate.config
\end{verbatim}
The available options for the CISM configuration file and 
for the GLINT climate interface configuration file are described in detail below.

Note that instructions for running CISM within the Community Earth System Model (CESM)
or another climate model are not described here.



\section{Overview of Configuration Files}

Running CISM is managed through configuration files (*.config) that enable 
desired model features and control input of initial conditions and/or forcing 
and output of model results.  This chapter summarizes the configuration options 
available for running CISM and is divided into sections on general Model Configuration, 
Input/Ouput Configuration, and optional Climate Forcing Configuration.

The format of CISM configuration files is taken from that used by the 
ConfigParser module in Python 2.x, which is similar to Windows \texttt{.ini} files 
and contains sections. Each section contains key, value(s) pairs.

\begin{itemize}
\item[Comments:] Empty lines, or lines starting with a \texttt{\#}, \texttt{;} or \texttt{!} are ignored.  Comments can also be added on the same line as a key/value pair using these delimiters.
\item[Sections:] A new section starts with the the section name enclosed with square brackets, \texttt{[]} and can be 20 characters long, e.g. \texttt{[grid]}.
\item[Key/Value Pairs:] Keys are separated from their associated values by a \texttt{=} or \texttt{:}. The names can be 20 characters long. Values can be 200 characters long.
\end{itemize}

Sections and keys are case sensitive and may contain white space. 
However, the configuration parser is very simple and thus the number of spaces 
within a key or section name also matters. Sensible defaults are used when 
a specific key is not found.  Defaults are indicated in bold in the tables below.

An example configuration file:
\begin{verbatim}
;a comment
[a section]
an_int  : 1
a_float = 2.0
a_char  = hey, this is rather cool
an_array = 10. 20. -10. 40. 100.

[another section]
! more comments
foo : bar
\end{verbatim}



