\section{Input/Output Configuration}

NetCDF I/O can be configured in the main configuration file or in a separate file 
(see \texttt{ioparams} in the \texttt{[options]} section). 
Any number of input, forcing, and output files can be specified. 

Input files are processed in the same order they occur in the configuration file, 
thus potentially overwriting previously loaded fields.  The configuration section 
for an input file specifies which time slice from the input file should be used as
the initial condition.  (Note this is an integer specifying the time level, not
the actual time.)

Input files can contain any of a large number of model fields that are specified as being
``loadable". These fields are marked with asterisks in the NetCDF model variable
table listed in Appendix \ref{ch:appendix_vars}. The option for a given variable to be ``loadable" 
or not can be changed within the \texttt{glide\_vars.def} file, as described in Appendix \ref{app:netcdfio}. 
The standard test cases discussed in Chapter \ref{chp:testcases} provide good examples for the types 
of variables that might be specified for standard model setups.    

Forcing files are new in CISM 2.0 and are input files that are read on every 
time step to allow time-dependent
forcing to be applied during a simulation.  Any inputtable fields can be included
in forcing files.  Forcing files should have a ``time'' field which is used to 
assign values to each field in the file during the simulation.  Forcing is applied 
in a piecewise constant fashion -- the most recent time slice in the forcing file prior
to the current model time is used on each time step.  (Linearly interpolatation of
forcing may be available in the future but is not currently implemented.)
Note that if a field is present in both an input file and in a
forcing file at the start time, the value in the forcing file will overwrite the value
from the input file because forcing files get read after input files.
Forcing files are processed in the same order they occur in the configuration file, 
on each time step thus potentially overwriting previously loaded fields from other
forcing files.  The ``dome'' test case (\texttt{tests/higher-order/dome}) includes 
an example of how to use time-dependent forcing.

One special input field is the \texttt{kinbcmask} field, which is used to specify locations (that is,
indices in the horizontal grid plane) on the staggered (velocity) grid where Dirichlet boundary 
conditions are to be applied. This input field is used in conjunction with the \texttt{uvel} and 
\texttt{vvel} fields; at any location where the value of \texttt{kinbcmask} is set to 1, the input field
values of \texttt{uvel} and \texttt{vvel} for that same column will be applied as Dirichlet boundary
conditions on the velocity solution. If \texttt{uvel} and \texttt{vvel} are not included in the input file, 
zero velocity will be specified throughout the column for any location where \texttt{kinbcmask} is 
set to 1. An example of this application can be seen in the confined shelf test case described 
in Chapter \ref{sc:ho-tests}. The construction of the relevant input \texttt{kinbcmask} field 
is done by the python script that constructs the other input fields for this test case 
(the \texttt{confined-shelf.py} script).  Also, the dome test case described in 
Chapter \ref{sc:ho-tests} includes a script and a config file demonstrating how 
\texttt{kinbcmask}, \texttt{uvel}, and \texttt{vvel} 
can be used in a forcing file to provide time-varying Dirichlet boundary conditions. 

\begin{center}
  \tablefirsthead{%
    \hline
  }
  \tablehead{%
    \hline
    \multicolumn{2}{|l|}{\emph{\small continued from previous page}}\\
    \hline
  }
  \tabletail{%
    \hline
    \multicolumn{2}{|r|}{\emph{\small continued on next page}}\\
    \hline}
  \tablelasttail{\hline}
  \begin{supertabular*}{\textwidth}{@{\extracolsep{\fill}}|l|p{11cm}|}
%%%% CF defaults
    \hline
    \multicolumn{2}{|l|}{\texttt{[CF default]}}\\
    \hline
    \multicolumn{2}{|p{0.95\textwidth}|}{This section contains metadata describing the experiment. Any of these parameters can be modified in the \texttt{[output]} section. The model automatically attaches a time stamp and the model version to the NetCDF output file.}\\
    \hline
    \texttt{title}& Title of the experiment\\
    \texttt{institution} & Institution at which the experiment was run\\
    \texttt{references} & References that might be useful\\
    \texttt{comment} & A comment, further describing the experiment\\
    \hline

%%%% CF Input
    \hline
    \hline
    \hline
    \multicolumn{2}{|l|}{\texttt{[CF input]}}\\
    \hline
    \multicolumn{2}{|p{0.95\textwidth}|}{Any number of input files can be specified in separate \texttt{[CF input]} sections. They are processed in the order they occur in the configuration file, potentially overriding previously loaded variables.}\\
    \hline
    \texttt{name}& The name of the NetCDF file to be read. Typically NetCDF files end with \texttt{.nc}.\\
    \texttt{time}& The time slice (not actual time) to be read from the NetCDF file. The first time slice is read by default.\\
    \hline

%%%% CF Forcing
    \hline
    \hline
    \hline
    \multicolumn{2}{|l|}{\texttt{[CF forcing]}}\\
    \hline
    \multicolumn{2}{|p{0.95\textwidth}|}{Any number of forcing files can be specified in separate \texttt{[CF forcing]} sections. They are processed in the order they occur in the configuration file, potentially overriding previously loaded variables.  Each forcing file needs a ``time'' dimension and variable that indicates the model time associated with each time slice in the file.}\\
    \hline
    \texttt{name}& The name of the NetCDF file to be read. Typically NetCDF files end with \texttt{.nc}.\\
    \hline

%%%% CF output
    \hline
    \hline
    \hline
    \multicolumn{2}{|l|}{\texttt{[CF output]}}\\
    \hline
    \multicolumn{2}{|p{0.95\textwidth}|}{This section of the NetCDF parameter file controls how often selected  variables are written to file.}\\
    \hline
    \texttt{name} & The name of the output NetCDF file. Typically NetCDF files end with \texttt{.nc}.\\
    \texttt{start} & Start writing to file when this time is reached (default: first time slice).\\
    \texttt{stop} & Stop writin to file when this time is reached (default: last time slice). \\
    \texttt{frequency} & The time interval in years, determining how often selected variables are written to file.\\
    \texttt{xtype} & Set the floating point representation used in NetCDF file. \texttt{xtype} can be one of \texttt{real}, \texttt{double} (default: \texttt{real}). \\
    \texttt{variables} & List of variables to be written to file. See Appendix \ref{ug.sec.varlist} for a list of known variables. Names should be separated by at least one space. The variable names are case sensitive. Variable \texttt{restart} selects all variables necessary for a restart.\\
    \hline
  \end{supertabular*}
\end{center}
