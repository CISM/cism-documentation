\subsection{Conservation of Momentum}

Starting from Newton's second law of motion, conservation of momentum is expressed as

\begin{equation}
\frac{d} {dt} \int_{V}\rho u_{i}~dV ~ = ~ \int_{V} \frac{\partial \sigma_{ij}} {\partial x_{j}} ~dV +  \int_{V} \rho g_{i}~dV
\label{eq:mobal1}
\end{equation}

\noindent
where $t$ is time, $\rho$ is density, $u$ is
velocity, $\sigma_{ij}$ is the stress tensor, $g$ is the
acceleration due to gravity, $V$ is the volume of an arbitrary
fluid element, and $(i,j)= \{x, y, z\}$ in a Cartesian coordinate
system. Equation \eqref{eq:mobal1} tells us that a fluid element of arbitrary size 
experiences a ``body force'' $\rho g_{i}\delta V$ due to gravity, which is balanced by 
stress divergence $\frac{\partial \sigma_{ij}} {\partial x_{j}} \delta V$ and acceleration
of the fluid in the volume $\delta V$. 

Making the assumptions that we have continuous fields and that ice is incompressible (i.e., 
its density $\rho$ does not change under conditions of interest), we can write

\begin{equation}
\rho \frac{D u_{i}}{D t}~=~\frac{\partial \sigma_{ij}}{\partial x_{j}} + \rho g_{i}
\label{eq:mobal2}
\end{equation}

\noindent
in which $D$ is a material derivative. Because the \href{http://en.wikipedia.org/wiki/Froude_number}
{Froude number} for ice flow is extremely small, the acceleration term (the first term on the left-hand side) can be 
neglected, leaving the steady-state form, 

\begin{equation}
\frac{\partial \sigma_{ij}}{\partial x_{j}} + \rho g_{i} ~=~0.
\label{eq:mobal3}
\end{equation}

Equation \eqref{eq:mobal3} states that the body force (the gravitational driving force) is balanced by forces resulting from 
gradients in the stress tensor $\sigma_{ij}$. All models of ice-flow dynamics are based on solving this equation in some form.
Chapters \ref{ch:glide} and \ref{ch:glissade} provide additional details on the approximations to this equation that are
solved by CISM.

The \href{http://en.wikipedia.org/wiki/Stress_tensor}{stress tensor} $\sigma_{ij}$ has nine components in a three-dimensional,
Cartesian coordinate system,

\begin{equation}
\mathbf{\sigma} =
\left\vert  \begin{array}{ccc} 
    \sigma _{ xx} & \sigma _{ xy} & \sigma _{ xz} \\
    \sigma _{ yx} & \sigma _{ yy} & \sigma _{ yz} \\
    \sigma _{ zx} & \sigma _{ zy} & \sigma _{ zz}. \\
\end{array} \right\vert 
\label{eq:mobal4}
\end{equation}

\noindent
Since $\sigma_{ij}$ is symmetric, only six of these components are independent.
The components along the diagonal are called normal stresses, and the off-diagonal components are called shear stresses. 
Deformation results not from the full stress but from the deviatoric stress,

\begin{equation}
\tau_{ ij} ~ = ~ \sigma _{ ij} ~ - ~{\frac{ 1}{ 3}} \sigma _{ kk} \delta _{ ij},
\label{eq:mobal5}
\end{equation}

\noindent
in which $\delta_{ ij}$ is the Kroneker delta (or the identify tensor). For shear stresses, \eqref{eq:mobal5} indicates that
the full and deviatoric stresses are identical.

\subsubsection{Constitutive relationship}

To relate the stress tensor to fluid motion, we introduce the strain rate tensor,

\begin{equation}
\dot{\epsilon}_{ij}~= \frac{1}{2}\left( \frac{ \partial u_{i}}{\partial x_{j}} + \frac{ \partial u_{j}}{\partial x_{i}}\right), ~~i,j = x,y,z,
\label{eq:mobal6}
\end{equation}

\noindent
where $u_i$ are the velocity vector components. The strain rate tensor $\dot{\epsilon}_{ij}$, and hence gradients in the 
velocity field, are related to the stress tensor $\tau_{ij}$ by a \href{http://en.wikipedia.org/wiki/Constitutive_equation}
{constitutive relation}. For a Newtonian fluid, this can be expressed as

\begin{equation}
\tau_{ij}~=\eta \dot{\epsilon}_{ij},
\label{eq:mobal7}
\end{equation}

\noindent
which states that the strain rate is proportional to the stress, with the ice viscosity $\eta$ serving as the constant of proportionality.
Ice does not behave as a Newtonian fluid, and instead exhibits a \href{http://en.wikipedia.org/wiki/Power-law_fluid}{power-law rheology}, 
so that it becomes more fluid (less viscous) the faster it deforms. This relationship can be expressed through Nye's generalization of Glen's 
flow law, 

\begin{equation}
\tau_{ij}~=~A(T^{*})^{\frac{-1}{n}} \dot{\epsilon}_{e}^{\frac{1-n}{n}} \dot{\epsilon}_{ij} 
\label{eq:mobal8}
\end{equation}

\noindent
in which $T^{*}$ is the absolute temperature corrected for the pressure dependence of the melt temperature, $\dot{\epsilon}_{e}$ is the 
second invariant (a norm) of the stress tensor, and the power-law exponent $n$ is commonly taken as 3. A comparison of 
\eqref{eq:mobal7} and \eqref{eq:mobal8} indicates that one can define an ``effective" ice viscosity for \eqref{eq:mobal7} as

\begin{equation}
\eta_{e}~=~A(T^{*})^{\frac{-1}{n}} \dot{\epsilon}_{e}^{\frac{1-n}{n}}.
\label{eq:mobal9}
\end{equation}

The temperature-dependent rate factor $A$ follows the Arrhenius relationship

\begin{equation}
A\left( T^{*}\right)~=~E A_{o}e^{-Q/RT^{*}},
\label{eq:mobal10}
\end{equation}

\noindent
in which $A_{o}$ is a constant, $Q$ is the activation energy for crystal creep, $R$ is the gas constant, and $E$ is a tuning 
parameter, which can be used to account for the effects of impurities and anisotropic ice fabrics. The homologous temperature is

\begin{equation}
T^{*}=T+\rho g H \Phi,
\label{eq:mobal11}
\end{equation}

\noindent
in which $\Phi$ is 9.8 $\times$10$^{-8}$ K Pa$^{-1}$, or about 8.7 $\times$10$^{-4}$ K m$^{-1}$ in ice. The pressure-dependent melt 
temperature is simply the triple point temperature minus the product $\rho g H \Phi$.
