\section{Conservation of Energy}

The first law of thermodynamics is used to make a basic statement of conservation of energy in a volume of ice $V$ enclosed within a surface $S$ is

\begin{equation}
\frac{d}{d t} \int_{V}E ~dV~=~- \int_{S}\mathbf{F}\cdot \hat{n}~dS~-~\int_{S}E \mathbf{u}\cdot \hat{n}~dS~+~\int_{V}W dV
\end{equation}

in which $E$ represents the energy of the volume, $F_{i}$ is the flux due
to diffusion, and $W$ represents any sources or sinks of energy within
the volume. The term $Eu_{i}$ is an energy flux through $S$ due to advection.
Following the steps laid out earlier, we use the divergence theorem and
the assumptions of continuous fields and incompressibility, such that

\begin{equation}
\frac{dE}{dt}~+~\nabla \cdot \left(F_{i} +E u_{i}  \right)~-~W~=~0
\end{equation}

Our goal is to use the first law of thermodynamics to compute the temperature of the ice and any changes in that temperature over time.

The energy $E$ is the product of density and the specific internal
energy of the ice $e$, which is itself the product of the specific heat
capacity $c_{p}$ and temperature $T$ because there is no transfer
between internal energy and pressure for an incompressible fluid. Thus,

\begin{equation}
\begin{matrix}
\frac{dE}{dt}&=&\frac{d\left(\rho e \right)}{dt} \\
&=&\rho\frac{de}{dt}~+~e \frac{d\rho}{dt}\\
&=&\rho c_{p} \frac{dT}{dt}
\end{matrix}
\end{equation}

The flux due to diffusion follows Fourier's ``law'' for heat conduction
so

\begin{equation}
\begin{matrix}
\nabla \cdot F_{i}&=&\nabla \cdot \left( -k ~\nabla T  \right) \\
&=&-k~\nabla^{2}T
\end{matrix}
\end{equation}

in which $k$ represents the thermal diffusivity of ice and we assume
gradients in its magnitude to be negligible.

Using progress made above and assuming that $\nabla \cdot u_{i}$ is
small with respect to other terms, we can write the advection term

\begin{equation}
\begin{matrix}
\nabla \cdot \left(E u_{i} \right)~=~\rho c_{p}~ u_{i} \cdot \nabla T  
\end{matrix}
\end{equation}

Two quantities must be considered as energy sources, the work done on
the system by internal deformation and the latent heat associated with
phase changes. The former is the product of strain rate and the
deviatoric stress $\dot{\epsilon}_{ij} \tau_{ij}$. The latter is the
product of the latent heat of fusion and the amount of material subject
to melting (freezing) per unit volume per unit time, $L_{f}M_{f}$.

At last, we are able to write equation
\textbackslash{}ref\{equation:enbal2\} in terms of temperature

\begin{equation}
\frac{\partial T}{\partial t}~=~\frac{k}{\rho c_{p}} \nabla^{2}T~-~u_{i}\cdot \nabla T~+~\frac{1}{\rho c_{p}} \dot{\epsilon}_{ij} \tau_{ij} ~+~\frac{1}{\rho c_{p}} L_{f} M_{f}
\end{equation}

It is often the case that horizontal terms
$\frac{\partial^{2} T}{\partial x^{2}}$ and
$\frac{\partial^{2} T}{\partial y^{2}}$ are small enough to be ignored.
