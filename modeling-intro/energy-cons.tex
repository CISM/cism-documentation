\subsection{Conservation of energy}

The first law of thermodynamics is used to make a basic statement of conservation of energy in a volume of ice $V$ enclosed within a surface $S$:

\begin{equation}
\frac{d}{d t} \int_{V}E ~dV~=~- \int_{S}\mathbf{F}\cdot \hat{n}~dS~-~\int_{S}E \mathbf{u}\cdot \hat{n}~dS~+~\int_{V}W dV
\label{eq:enbal1}
\end{equation}

\noindent
in which $E$ is the total energy within the volume, $F_{i}$ is the energy flux due
to diffusion, and $W$ represents any sources or sinks of energy within
the volume. The term $Eu_{i}$ is an energy flux through $S$ due to advection.
Following the steps laid out earlier, we use the divergence theorem and
the assumptions of continuous fields and incompressibility to obtain

\begin{equation}
\frac{dE}{dt}~+~\nabla \cdot \left(F_{i} +E u_{i}  \right)~-~W~=~0
\label{eq:enbal2}
\end{equation}

\noindent
Our goal is to use the first law of thermodynamics to compute the temperature of the ice and any changes in that temperature over time.

The energy $E$ is the product of density and the specific internal
energy of the ice $e$, which is itself the product of the specific heat
capacity $c_{p}$ and temperature $T$ (because there is no transfer
between internal energy and pressure for an incompressible fluid). Thus,

\begin{equation}
\begin{matrix}
\frac{dE}{dt}&=&\frac{d\left(\rho e \right)}{dt} \\
&=&\rho\frac{de}{dt}~+~e \frac{d\rho}{dt}\\
&=&\rho c_{p} \frac{dT}{dt}
\end{matrix}
\label{eq:enbal3}
\end{equation}

\noindent
The heat flux due to diffusion follows Fourier's ``law'' for heat conduction,

\begin{equation}
\begin{matrix}
\nabla \cdot F_{i}&=&\nabla \cdot \left( -k ~\nabla T  \right) \\
&=&-k~\nabla^{2}T,
\end{matrix}
\label{eq:enbal4}
\end{equation}

\noindent
in which $k$ is the thermal conductivity of ice and we assume
gradients in its magnitude to be negligible. 

Using progress made above, we can write the advection term

\begin{equation}
\begin{matrix}
\nabla \cdot \left(E u_{i} \right)~=~\rho c_{p}~ u_{i} \cdot \nabla T. 
\end{matrix}
\label{eq:enbal5}
\end{equation}

\noindent
In the expansion of the terms on the left-hand side of \eqref{eq:enbal5} (using the product rule), we 
have implicitly ignored the term involving $\nabla \cdot u_{i}$ because it is small with respect to the other terms
retained on the right-hand side.

Two energy sources must be considered: the work done on
the system by internal deformation and the latent heat associated with
phase changes. The former is the product of the strain rate and
deviatoric stress, $\dot{\epsilon}_{ij} \tau_{ij}$. The latter is the
product of the latent heat of fusion and the amount of material (ice) subject
to melting (or freezing) per unit volume, per unit time, $L_{f}M_{f}$.

At last, we can write equation \eqref{eq:enbal2} in terms of temperature:

\begin{equation}
\frac{\partial T}{\partial t}~=~\frac{k}{\rho c_{p}} \nabla^{2}T~-~u_{i}\cdot \nabla T~+~\frac{1}{\rho c_{p}} \dot{\epsilon}_{ij} \tau_{ij} ~+~\frac{1}{\rho c_{p}} L_{f} M_{f}.
\label{eq:enbal6}
\end{equation}

\noindent
It is often the case that horizontal terms
$\frac{\partial^{2} T}{\partial x^{2}}$ and
$\frac{\partial^{2} T}{\partial y^{2}}$ are small enough to be ignored.
