%%%%%%%%%%%%%%%%%%%%%%%%%%%%%%%%%%%%%%%%%%%%%%%%%%%%%%%%%
% This is a new description of the temperature solver
% Not yet integrated into the main docs
%%%%%%%%%%%%%%%%%%%%%%%%%%%%%%%%%%%%%%%%%%%%%%%%%%%%%%%%%

\documentclass[10pt,english,a4paper]{article}
\usepackage[intlimits]{amsmath}
\usepackage{amsthm,natbib}
\newcommand{\diffn}[2]{\ensuremath{\frac{\partial #1}{\partial #2}}}
\newcommand{\diffnsq}[2]{\ensuremath{\frac{\partial^2 #1}{\partial #2^2}}}
\newcommand{\diffnx}[3]{\ensuremath{\frac{\partial^2 #1}{\partial #2
      \partial #3}}}

\begin{document}
\title{GLIDE Temperature Solver}
\author{Ian Rutt}
\maketitle

\section{Basic equation}
%
The basic temperature equation to be solved is:
%
\begin{equation}
  \frac{\partial T}{\partial t} = \frac{k}{\rho_I c_I}\left(\nabla^2 T
  +\frac{\partial^2 T}{\partial z^2}\right)
  -\mathbf{u}\cdot\nabla T
  -w\frac{\partial T}{\partial z}
  +\frac{\Phi}{\rho_I c_I}
\end{equation}
%
The first term on the lefthand side is conduction, the second and
third are horizontal advection, the fourth is heat production due to
deformation. A very important simplification is to ignore horizontal
diffusion of heat (the $\nabla^2 T$ term):
%
\begin{equation}
  \frac{\partial T}{\partial t} = \frac{k}{\rho_I c_I}\frac{\partial^2 T}{\partial z^2}
  -\mathbf{u}\cdot\nabla T
  -w\frac{\partial T}{\partial z}
  +\frac{\Phi}{\rho_I c_I}
\label{eqn.temp_eqn}
\end{equation}
%
\section{Coordinate transformation}
%
To solve these equations, we transform the coordinate system
$(x,y,z,t)\rightarrow (x',y',\sigma,t')$, with $\sigma$ defined
%
\begin{equation}
\sigma = \frac{s-z}{H}.
\label{eqn.defn_sigma}
\end{equation}
%
Applying the methods of \citet{Hindmarsh1988}, we obtain the following
expressions for the various derivatives in the new coordinate system:
%
\begin{eqnarray}
\diffn{}{x} &=& \diffn{}{x'}+\frac{1}{H}\left(\diffn{s}{x}-\sigma\diffn{H}{x}\right)\diffn{}{\sigma}\\
\diffn{}{y} &=& \diffn{}{y'}+\frac{1}{H}\left(\diffn{s}{y}-\sigma\diffn{H}{y}\right)\diffn{}{\sigma}\\
\diffn{}{t} &=& \diffn{}{t'}+\frac{1}{H}\left(\diffn{s}{t}-\sigma\diffn{H}{t}\right)\diffn{}{\sigma}\\
\diffn{}{z} &=& -\frac{1}{H}\diffn{}{\sigma}\\
\diffnsq{}{z} &=& \frac{1}{H^2}\diffnsq{}{\sigma}\\
\end{eqnarray}
%
Using these expressions, we write the transformed temperature equation
(\ref{eqn.temp_eqn}) as follows:
%
\begin{multline}
\diffn{T}{t'}+\frac{1}{H}\left(\diffn{s}{t}-\sigma\diffn{H}{t}\right)\diffn{T}{\sigma}
= \frac{k}{\rho_I c_I H^2}\diffnsq{T}{\sigma} -
\mathbf{u}\cdot\hat{\nabla}T\\ 
- \left(\frac{\mathbf{u}\cdot \nabla s
  -\sigma \mathbf{u}\cdot\nabla H}{H}\right)\diffn{T}{\sigma} +\frac{w}{H}\diffn{T}{\sigma}+\frac{\Phi}{\rho_I c_I},
\end{multline}
%
with $\hat{\nabla}$ being the derivative on $(x',y')$. Grouping
together terms in $\partial T/\partial \sigma$, we can write:
%
\begin{multline}
\diffn{T}{t'} = \frac{k}{\rho_I c_I H^2}\diffnsq{T}{\sigma} -
\mathbf{u}\cdot\hat{\nabla}T\\
+\frac{1}{H}\left[w-\diffn{s}{t}-\mathbf{u}\cdot \nabla s + \sigma
\left( \diffn{H}{t} +\mathbf{u}\cdot\nabla H \right) \right]\diffn{T}{\sigma}+\frac{\Phi}{\rho_I c_I}.
\end{multline}
%
We can define a vertical grid velocity $w_{\mathrm{grid}}$,
%
\begin{equation}
w_{\mathrm{grid}}=\diffn{s}{t}+\mathbf{u}\cdot \nabla s - \sigma
\left( \diffn{H}{t} +\mathbf{u}\cdot\nabla H \right),
\end{equation}
%
and an effective vertical velocity $w_{\mathrm{eff}}=w-w_{\mathrm{grid}}$, allowing the temperature equation to be simplified as follows:
%
\begin{equation}
\frac{\partial T}{\partial t'} = \frac{k}{\rho_I c_I
  H^2}\frac{\partial^2 T}{\partial \sigma^2}
  - \mathbf{u}\cdot\hat{\nabla}T
  + \frac{w_{\mathrm{eff}}}{H}\frac{\partial T}{\partial
  \sigma}+\frac{\Phi}{\rho_I c_I}, 
\label{eqn.simple_temp}
\end{equation}
%
\section{Heat-generation term}
%
The expression for the heat-generation term is
%
\begin{equation}
\Phi=\sum_{ij}\dot{\varepsilon}_{ij}\tau_{ij}.
\end{equation}
%
Neglecting higher-order stresses, this becomes:
%
\begin{equation}
\Phi = 2\dot{\varepsilon}_{xz}\tau_{xz}+2\dot{\varepsilon}_{yz}\tau_{yz}
\end{equation}
%
One approach to solving this is to substitute for the stress and strain-rates
separately, which yields
%
\begin{equation}
\Phi = -\rho_I g(s-z)\diffn{\mathbf{u}}{z}\cdot\nabla s.
\end{equation}
%
{\bf However, this is not what is done in Glimmer.} Instead, we use
Glen's Flow Law:
%
\begin{equation}
\dot{\varepsilon}_{iz}=A(T^*)\tau_{*}^{(n-1)}\tau_{iz},
\end{equation}
%
where $\tau_{*}$ is the effective stress, which for the SIA is given
by
%
\begin{equation}
\tau_{*}=\left(\tau_{xz}^2+\tau_{yz}^2\right)^{\frac{1}{2}}.
\end{equation}
%
This means that the heat-generation term may be written as
%
\begin{equation}
\frac{\Phi}{\rho_I c_I}=\frac{2A(T^*)}{\rho_I
  c_I}\left(\tau_{xz}^2+\tau_{yz}^2\right)^{\frac{n+1}{2}}.
\label{eqn.heat_gen_stress}
\end{equation}
%
The stresses are known from the SIA: $\tau_{xz}$ is given by 
%
\begin{equation}
\tau_{xz}=-\rho g(s-z)\diffn{s}{x},
\end{equation}
%
with a similar expression for $\tau_{yz}$. Transforming to
$\sigma$-coordinates, we have
%
\begin{equation}
\tau_{xz}=-\rho g\sigma H\diffn{s}{x}.
\label{eqn.SIA_stress}
\end{equation}
%
Substituting into (\ref{eqn.heat_gen_stress}), yields
%
\begin{equation}
\frac{\Phi}{\rho_I c_I}=\frac{2A(\rho g \sigma H)^{n+1}}{\rho_I
  c_I}\left[\left(\diffn{s}{x}\right)^2+\left(\diffn{s}{y}\right)^2\right]^{\frac{n+1}{2}}.
\end{equation}
%
(Note that the negative sign in (\ref{eqn.SIA_stress}) disappears
because the squares of the stresses in (\ref{eqn.heat_gen_stress}) are
all positive).

Finally, substituting into
(\ref{eqn.simple_temp}), we get:
%
\begin{multline}
\frac{\partial T}{\partial t'} = \frac{k}{\rho_I c_I
  H^2}\frac{\partial^2 T}{\partial \sigma^2}
  - \mathbf{u}\cdot\hat{\nabla}T
  + \frac{w_{\mathrm{eff}}}{H}\frac{\partial T}{\partial
  \sigma}\\
+\frac{2A(\rho g \sigma H)^{n+1}}{\rho_I
  c_I}\left[\left(\diffn{s}{x}\right)^2+\left(\diffn{s}{y}\right)^2\right]^{\frac{n+1}{2}}. 
\label{eqn.final_temp}
\end{multline}
%
\section{Discretization}
%
\subsection{Vertical discretisation}
%
The vertical grid is irregular, so define
%
\begin{eqnarray}
\Delta\sigma_{+} & = & \sigma_{i+1}-\sigma_{i}\\
\Delta\sigma_{-} & = & \sigma_{i}-\sigma_{i-1}\\
\Delta\sigma_{0} & = & \sigma_{i+1}-\sigma_{i-1}
\end{eqnarray}
%
First-order centred difference (first-order accurate):
%
\begin{equation}
\frac{\partial T}{\partial \sigma} = \frac{T_{i+1}-T_{i-1}}{\Delta \sigma_0}
\end{equation}
%
Second-order centred difference (second-order accurate):
%
\begin{equation}
\frac{1}{2}\frac{\partial^2 T}{\partial \sigma^2} = \frac{T_{i+1}}{\Delta \sigma_{+}\Delta \sigma_{0}}
-\frac{T_{i}}{\Delta \sigma_{-}\Delta \sigma_{+}}
+\frac{T_{i-1}}{\Delta \sigma_{-}\Delta \sigma_{0}}
\end{equation}
%
\subsection{Temporal discretization}
%
The temporal discretization uses a simple forward-difference. Parts of
the equation have implicit temporal discretisation.
%
\begin{equation}
\frac{\partial T}{\partial t}=\frac{T_{i}^{n+1}-T_{i}^{n}}{\Delta t}
\end{equation}
%
\subsection{Horizontal discretization}
%
The advective terms are discretized using a second-order up-winding
scheme. In determining which direction to use, the surface velocity is
tested, so that for $u(\sigma=0)>0$:
%
\begin{equation}
\diffn{T}{x}=\frac{3T_{i,j}-4T_{i-1,j}+T_{i-2,j}}{2\Delta x},
\end{equation}
%
and for $u(\sigma=0)<0$:
%
\begin{equation}
\diffn{T}{x}=\frac{-T_{i+2,j}+4T_{i+1,j}-3T_{i,j}}{2\Delta x}.
\end{equation}
%
For notational purposes in the following analysis, we combine these as
%
\begin{equation}
\left.\diffn{T}{x}\right|_{i,j}=\frac{1}{2\Delta x}\sum_{m=-2}^{2}\alpha_{m}T_{i+m,j},
\end{equation}
%
and, for $y$
%
\begin{equation}
\left.\diffn{T}{y}\right|_{i,j}=\frac{1}{2\Delta y}\sum_{m=-2}^{2}\beta_{m}T_{i,j+m}.
\end{equation}
%
\section{Combining things together}
%
Substituting the discrete forms:
%
\begin{multline}
\frac{T_{i}^{n+1}-T_{i}^{n}}{\Delta t}=\frac{2k}{\rho c H^2}\left[\frac{T_{i+1}}{\Delta \sigma_{+}\Delta \sigma_{0}}
-\frac{T_{i}}{\Delta \sigma_{-}\Delta \sigma_{+}}
+\frac{T_{i-1}}{\Delta \sigma_{-}\Delta \sigma_{0}}\right]\\
+\frac{w_{\mathrm{eff}}}{H}\left[\frac{T_{i+1}-T_{i-1}}{\Delta \sigma_0}\right]
\end{multline}
%
Solve semi-implicitly, so that $T_i=\frac{1}{2}(T_i^{n+1}+T_i^{n})$.
%
\begin{multline}
\frac{T_{i}^{n+1}-T_{i}^{n}}{\Delta t}=
\frac{k}{\rho c H^2}\left[\frac{T^{n+1}_{i+1}+T^{n}_{i+1}}{\Delta \sigma_{+}\Delta \sigma_{0}}
-\frac{T^{n+1}_{i}+T^{n}_{i}}{\Delta \sigma_{-}\Delta \sigma_{+}}
+\frac{T^{n+1}_{i-1}+T^n_{i-1}}{\Delta \sigma_{-}\Delta \sigma_{0}}\right]\\
+\frac{w_{\mathrm{eff}}}{2H}\left[\frac{T^{n+1}_{i+1}+T^{n}_{i+1}-T^{n+1}_{i-1}-T^{n}_{i-1}}{\Delta \sigma_0}\right]
\end{multline}
%
Collecting terms of knowns and unknowns:
%
\begin{multline}
\left[-\frac{k\Delta t}{\rho c H^2}\frac{1}{\Delta\sigma_{+}\Delta\sigma_{0}}-\frac{w_{\mathrm{eff}}\Delta t}{2H\Delta\sigma_0}\right]T^{n+1}_{i+1}
+\left[1+\frac{k\Delta t}{\rho c H^2}\frac{1}{\Delta\sigma_{-}\Delta\sigma_{+}}\right]T^{n+1}_i\\
+\left[-\frac{k\Delta t}{\rho c H^2}\frac{1}{\Delta\sigma_{-}\Delta\sigma_{0}}+\frac{w_{\mathrm{eff}}\Delta t}{2H\Delta\sigma_0}\right]T^{n+1}_{i-1}=
\left[\frac{k\Delta t}{\rho c H^2}\frac{1}{\Delta\sigma_{+}\Delta\sigma_{0}}+\frac{w_{\mathrm{eff}}\Delta t}{2H\Delta\sigma_0}\right]T^{n}_{i+1}\\
+\left[1-\frac{k\Delta t}{\rho c H^2}\frac{1}{\Delta\sigma_{-}\Delta\sigma_{+}}\right]T^{n}_i
+\left[\frac{k\Delta t}{\rho c H^2}\frac{1}{\Delta\sigma_{-}\Delta\sigma_{0}}-\frac{w_{\mathrm{eff}}\Delta t}{2H\Delta\sigma_0}\right]T^{n}_{i-1}
\end{multline}
%
The middle term of the second-order differential is split (I think this is for computational efficiency), so that:
%
\begin{equation}
\frac{1}{\Delta\sigma_{-}\Delta\sigma_{+}}=\frac{1}{\Delta\sigma_{0}\Delta\sigma_{+}}+\frac{1}{\Delta\sigma_{0}\Delta\sigma_{-}}
\end{equation}
%
Giving finally:
%
\begin{multline}
\left[-\frac{k\Delta t}{\rho c H^2}\frac{1}{\Delta\sigma_{+}\Delta\sigma_{0}}-\frac{w_{\mathrm{eff}}\Delta t}{2H\Delta\sigma_0}\right]T^{n+1}_{i+1}
+\left[1+\frac{k\Delta t}{\rho c H^2}\left(\frac{1}{\Delta\sigma_{0}\Delta\sigma_{+}}+\frac{1}{\Delta\sigma_{0}\Delta\sigma_{-}}\right)\right]T^{n+1}_i\\
+\left[-\frac{k\Delta t}{\rho c H^2}\frac{1}{\Delta\sigma_{-}\Delta\sigma_{0}}+\frac{w_{\mathrm{eff}}\Delta t}{2H\Delta\sigma_0}\right]T^{n+1}_{i-1}=
\left[\frac{k\Delta t}{\rho c H^2}\frac{1}{\Delta\sigma_{+}\Delta\sigma_{0}}+\frac{w_{\mathrm{eff}}\Delta t}{2H\Delta\sigma_0}\right]T^{n}_{i+1}\\
+\left[1-\frac{k\Delta t}{\rho c H^2}\left(\frac{1}{\Delta\sigma_{0}\Delta\sigma_{+}}+\frac{1}{\Delta\sigma_{0}\Delta\sigma_{-}}\right)\right]T^{n}_i
+\left[\frac{k\Delta t}{\rho c H^2}\frac{1}{\Delta\sigma_{-}\Delta\sigma_{0}}-\frac{w_{\mathrm{eff}}\Delta t}{2H\Delta\sigma_0}\right]T^{n}_{i-1}
\end{multline}
%
\begin{thebibliography}{99}
\bibitem[Hindmarsh and Hutter(1988)Hindmarsh and
  Hutter]{Hindmarsh1988}
  Hindmarsh R.C.A., Hutter K. (1988) Numerical fixed domain mapping
  solution of free-surface flows coupled with an evolving interior
  field. {\it Int. J. for Numerical and Analytical Methods in
  Geomechanics}, {\bf 12}, 437--459
\end{thebibliography}
\end{document}
