

\section{Introduction}
\label{sc:glissade-intro}

CISM includes a higher-order dynamical core called Glissade.  (The name ``Glissade'' was originally an acronym,
but as with ``Glimmer'' the acronym is rarely used.) Glissade is the successor to the shallow-ice
Glide dycore in the original Glimmer code.  Like Glide, Glissade solves equations for conservation of momentum
(i.e., an appropriate approximation of Stokes flow), mass, and internal energy. Glissade numerics, however,
differ substantially from Glide numerics:

\begin{itemize}

\item \textit{Velocity:}
Glide is limited to the shallow-ice approximation (SIA), whereas Glissade can solve several Stokes
approximations, as described in Section \ref{sc:glissade-velocity}.
These include a 3D first-order Blatter-Pattyn solver (the most accurate and complex scheme)
as well as simpler shallow-ice and shallow-shelf solvers.

\item \textit{Temperature:}
To evolve the ice temperature, Glide solves a prognostic equation that incorporates horizontal advection as well
as vertical heat diffusion and internal dissipation.  In Glissade, temperature advection is handled by the
transport scheme, and a separate module solves for vertical diffusion and internal dissipation.
The vertical solver, described in Section \ref{sc:glissade-temperature}, is local; 
each ice column calculation is independent of other column calculations.

\item \textit{Mass and tracer transport:}
Glide solves a diffusion equation for tranport of mass (i.e., thickness); this equation incorporates the local shallow-ice
velocities. Since Glissade solves for higher-order flow that may have a large advective (as oppposed to diffusive)
component, a different approach is needed.  Glissade has two mass- and energy-conserving transport options, 
described in Section \ref{sc:glissade-transport}:
incremental remapping (the more complex and accurate scheme) and first-order upwind. These schemes
transfer mass and internal energy in the horizontal direction, followed by a vertical remapping to
the sigma coordinate system.

\end{itemize}

Glissade numerics are described in detail below.


%Some introductory text that might be useful here, but that was out of place in a previous section 
%(taken from the end of ./modeling-intro/modeling-intro.tex), has been pasted into the respective .tex file
%for this section (./glissade/glissade-intro.tex) but is commented out for now.


%3D, thermo-mechanically coupled ice sheet models. 3D refers to the
%explicit vertical layering of the model for computing temperature.
%Thermo-mechanical means that the ice viscosity is sensitive to
%temperature, and an iterative procedure must be used to find the flow
%rates from temperature. Both models exploit the often used shallow ice
%approximation \textbackslash{}citep\{hutter83\}, which accounts for the
%membrane like nature of ice sheets by reducing the stress tensor to only
%leading order terms resulting from simple shear. The shallow ice
%approximation, when combined with the non-linear constituative relation
%given by Glen's flow law \textbackslash{}citep\{paterson94\}, leads to
%the following expression for horizontal velocities
%
%$\begin{matrix}
%u_i(z) &=& -2 (\rho g)^n \vert \nabla s\vert ^{n-1} \frac{\partial s}{\partial i} 
%\int_h^z A(\theta^*)(s-z)^n dz + u_i(h)\\
%i &=& x,y.
%\end{matrix}$
%
%All parameters and symbols used in this paper appear in table
%\textbackslash{}ref\{symbols\}. It is worth noting that all quantities
%used to find the horizontal velocities are computed locally. The shallow
%ice approximation eliminates terms resulting from transverse or lateral
%stresses. The temperature sensitivity of ice flow is given by an
%Arrhenius relation,
%
%$
%A(\theta^*) = a \exp \left(\frac{-Q}{R\theta^*}\right).
%$
%
%Which, in typical ice temperature ranges (-50 -{}- 0 C$^\circ$) varies
%over 3 orders of magnitude. However because most shear occurs at the
%base, and the base is warmed by dissipation and geothermal heat flow, a
%more appropriate range is -20 -{}- 0 C$^\circ$. This gives a range of
%$A$ over about a factor of 30.
%
%Assuming that horizontal diffusion is negligible, again due to the
%membrane nature or very small aspect ratio (ratio of vertical to lateral
%extent) of ice sheets, the ice temperature field is found from the
%conservation of energy
%
%$ 
%\frac{\partial \theta}{\partial t} = \frac{k}{\rho c}
%\frac{\partial^2
%\theta}{\partial z^2} -
%\mathbf{u} \cdot \nabla \theta -
%u_z \frac{\partial \theta}{\partial z} 
%+ \frac{g(s-z)}{c}\frac{\partial \mathbf{u}}{\partial z} \cdot \nabla s.
%$
%
%The terms on the right hand side from left to right are vertical
%diffusion, horizontal advection, vertical advection, and dissipation
%(using only the shallow ice stress tensor). Equation
%\textbackslash{}ref\{temp\} is subject to the boundary conditions
%
%$\begin{matrix}
%\theta - \theta_s(x,y) = & 0 &~\forall z=s \\
%k \nabla \theta \cdot \mathbf{\hat n}(h) =& -G(x,y) + \mathbf{u} \cdot \tau_d
%&~\forall z=h.
%\end{matrix}$
%
%The upper surface is set to a mean annual temperature ($\theta_s$), and
%the lower surface is accounting for the heat sources from both
%geothermal heat ($G$) flux, and frictional heat generated when ice
%slides over the bed. The temperatures are constrained by the melting
%point corrected for pressure via the Clausius-{}-Clapeyron gradient
%
%$
%\theta^* = \theta - \beta (s-z)
%$
%
%The vertical advection term in equation \textbackslash{}ref\{temp\}
%requires vertical velocities. They are found from incompressibility,
%
%$
%\frac{\partial u_x}{\partial x} + 
%\frac{\partial u_y}{\partial y} +
%\frac{\partial u_z}{\partial z} = 0,
%$
%
%by integrating with respect to $z$, giving
%
%$
%u_z(z) = -\int_h^z \left( \frac{\partial u_x}{\partial x} + \frac{\partial
%u_y}{\partial y}\right ) dz + M + \mathbf{u}(h) \cdot \nabla h.
%$
%
%Were the complete accounting must include the melt rate and bed
%topography. Basal melt rates computed from the jump boundary condition
%at the bed,
%
%$
%M = \frac{1}{\rho L} \left ( k \frac{\partial
%\theta(h)}{\partial z} + G + \mathbf{u} \cdot \tau_d \right ).
%$
%
%Having solved for the temperature dependent velocity fields, changes in
%the ice sheet's geometry are computed from the continuity equation
%
%$
%\frac{\partial H}{\partial t} = - \nabla \cdot (\mathbf{\bar u} H) + B - M.
%$
%
%Both models use the finite difference methods for descritization of the
%partial differential equations. GLIMMER differs from PISM in that it
%offers a choice of implicit schemes for solving equation
%\textbackslash{}ref\{continuity\}, whereas PISM uses an explicit scheme
%\textbackslash{}citep\{press92\}. Further, GLIMMER utilizes a rescaled,
%or $\sigma$ vertical coordinate \textbackslash{}citep\{lliboutry87\} and
%PISM does not. PISM uses the PETSc
%\textbackslash{}citep\{petsc-user-ref\} library to achieve parallelism
%and has the capacity to include a more complete stress formulation, but
%that capacity was not used here. Additional discussion of the field
%equations and numerical methods used in GLIMMER can be found in
%\textbackslash{}citet\{payne97\} and \textbackslash{}citet\{payne99\}.
%
%The ice sheet model CISM uses a finite difference method to solve the
%governing thermodynamic equations for ice using the \{\textbackslash{}it
%shallow ice approximation\}. This is the approach generally adopted for
%modeling large ice masses. The assumption is made that slopes at the
%upper and lower surfaces are sufficiently small that normal stress
%components can be neglected. This leads to a \{\textbackslash{}it
%local\} balance between the gravitational driving stress and the basal
%shear stress and expressions for the shear stresses
%
%$\begin{matrix}
%\tau_{xz}(z)&=&-\rho g \left(s - z \right) \frac{\partial s}{\partial x}  \\
%\tau_{yz}(z)&=&-\rho g \left(s - z \right) \frac{\partial s}{\partial y}
%\end{matrix}$
%
%Evolution of the ice thickness uses equation
%\textbackslash{}ref\{equation:mabalfinc\} and the temperature solver
%uses a version of equation \textbackslash{}ref\{equation:enbalfin\}
%simplified to neglect horizontal diffusion (a typical simplification).
%
%The model equations are solved on a regular grid using the Glen flow law
%(equation \textbackslash{}ref\{equation:Glen\}) and appropriate boundary
%conditions for the upper and lower surfaces. These include the surface
%ice accumulation rate and temperature and a geothermal gradient (applied
%at the base of a bedrock layer with specified thermal properties). Basal
%traction may also be specified, in the situation where ice is at the
%melt temperature at the base. Isostatic adjustment of the land surface
%beneath the ice sheet, not discussed here, is also included.
%
%\subsubsection{numerical scheme}
%
%The continuous functions represented by the model governing equations
%cannot be solved exactly. Instead, they are discretized so that finite
%approximations of their solutions may be made. There are a variety of
%numerical techniques available for this purpose, GLIMMER makes use of a
%finite difference method.
%
%In brief, the model domain (a region of Earth's surface, for example,
%Greenland) is subdivided into a regularly-spaced horizontal grid and
%derivatives are approximated along the grid directions. The grid is
%fixed in space over the course of the model run. Model variables such as
%ice thickness are updated at each time step according to the numerical
%approximations of the governing equations. The vertical dimension is
%treated using a non-dimensional ``stretch'' coordinate so that an
%evolving ice thickness may be accommodated. The scaling is:
%
%$
%\zeta~=~\frac{s-z}{H}
%$
%
%so that $\zeta=1$ at the surface $s$ and $\zeta=0$ at the base. The
%governing equations must be re-written in the new, $(x, y, \zeta)$
%coordinate system.
%
%If you would like to read more about the inner workings of GLIMMER, its
%documentation is available at the class website. This is not necessary
%for the present lab exercise.



