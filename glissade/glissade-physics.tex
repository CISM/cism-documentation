
\section{Other model physics}
\label{glissade-physics}
The Glissade modules for velocity, temperature, transport, and calving replace the
corresponding Glide modules when \texttt{dycore} = 2 is set in the config file.
Other model physics, however, remains similar to Glide.  In particular:

\begin{itemize}

%TODO - Write a new section on calving.
%TODO - Write a new section on basal physics options.

\item Isostasy is treated in the same way as in Glide (see Section~\ref{sc:glide_isos}).
The lithosphere can be treated as local or elastic, and the asthenosphere is
either fluid or relaxing.  The elastic lithosphere calculation
is non-local and has not been parallelized.  Thus it is not possible to simulate
an elastic lithosphere when running on more than one processor.

\item Typically, the geothermal heat flux is set to a constant or prescribed from
an input file.  As in Glide, however, it is also possible to compute the
geothermal heat flux from a model of heat flow in bedrock (see Section~\ref{sc:glide_thermal_bc}).
This model includes non-local horizontal diffusion which, like the elastic
lithosphere, has not been parallelized and therefore will not work on multiple
processors.

\item The basal water options in Glissade are also the same as those in Glide.
These are all local except for \texttt{basal\_water} = 2, which computes a flux
of basal water based on a steady-state routing calculation.  The routing calculation
is not parallel and has not been tested with Glissade.

\end{itemize}

A sophisticated basal hydrology model is under development and will likely be
included in future versions of CISM.  This model will output the effective pressure,
which is a required input for the Coulomb friction law
(\texttt{which\_ho\_babc} = 10) recently introduced in CISM.
For now the effective pressure is either read from an input file
or parameterized simply (using \texttt{basal\_water} = 4).
The Coulomb friction law is not yet scientifically supported and
should be used at your own risk.

In summary, although CISM has a sophisticated higher-order dycore,
the physics parameterizations remain fairly simple, especially for runs
with multiple processors.  This is an area of active development.

