
\section{Calving}
\label{sc:glissade-calving}

Glide includes several simple schemes for iceberg calving.
Three of these schemes are supported in Glissade:

\begin{itemize}
\item Calve all floating ice.
\item Calve ice where the current bedrock lies below a certain elevation.
\item Calve ice where the relaxed bedrock (i.e., the bedrock toward which
a viscous asthenosphere is relaxing) lies below a certain elevation.
\end{itemize}

In addition, there is a more complex scheme to calve floating ice thinner than a user-defined threshold $H_C$.
As discussed by \citet{Albrecht2011}, this is not as simple as it sounds.
Suppose $H_C = 100 \mathrm{~m}$, and an ice shelf with ice thicker than 100~m is advancing.
During the next time step, ice will be transported to grid cells just downstream of the initial calving front. 
Provided the time step is not too long, these cells will have $H < H_C$.
If this thin ice immediately calves, then the calving front cannot advance.

Glissade avoids this problem by defining a protected ring of thin ice at the margin of floating ice.
If ice in a grid cell has $H < H_C$ but shares an edge with a cell where $H \geq H_C$, then it is protected from calving.
It can then thicken due to transport from neighboring cells, while remaining dynamically inactive. 
Once the ice in this cell thickens to $H \geq H_C$, it becomes dynamically active, 
and its thin-ice neighbors become part of the protected ring.

Suppose, on the other hand, that ice is thinning at the calving front. A cell might initially
have $H \geq H_C$, with one or more protected thin-ice neighbors. If the ice in this cell 
thins to $H < H_C$, it becomes a dynamically inactive cell in the protected ring.  Its thin-ice neighbors 
(assuming they have no other neighbors with $H \geq H_C$) are no longer protected and are allowed to calve.
In this way the calving front can retreat.

The thickness-based calving scheme is more realistic than the simple schemes listed above, but
still is likely to give large errors in calving-front location.
A more accurate damage-based scheme is under development and will be included in future versions of CISM.
In this scheme, a scalar damage parameter evolves as the ice undergoes strain, and ice calves
when the damage in a grid cell exceeds a prescribed threshold.

By default, any floating ice that meets the calving criterion (e.g., it is thin and lies outside the protected ring,
or the bedrock topography lies below the threshold elevation) will calve on a given time step.
A user might prefer, however, to limit the extent of calving.
By changing the value of \texttt{calving\_domain}, the user can specify that ice calves 
(1) only when it lies at the marine margin (i.e., it borders an ice-free ocean cell), or 
(2) only when it is connected to the ocean through other cells that meet the calving criterion.
In this way, cells in the interior of an ice shelf can be prevented from calving.

The user can also prescribe a calving timescale. If this timescale is longer than the model time step,
then only a fraction of the ice that meets the calving criterion will calve in a single time step.
For example, if the model time step is 1 yr and the calving timescale is 2 yr, then half the
ice thickness in each calving-eligible cell will calve.

A related option is to remove so-called floating ice islands.  A floating island is a connected region
where every cell is floating and thus has no basal traction.  These islands can be present in initial
data sets or can arise occasionally during simulations, depending on the
calving scheme.  Finding the velocity for such a region is an ill-posed problem, so it is best
to remove floating islands as soon as they occur. This is done by first marking all the grounded ice cells,
and then using a boundary-fill algorithm to mark every cell that is connected to a grounded cell.
(That is, one could travel from a grounded cell to the cell in question along a path at least
one cell wide, without leaving the ice pack.) For parallel simulations, the algorithm is repeated enough times
to connect cells on all processors. Any floating cells remaining are ice islands that immediately calve.

When CISM is coupled to a climate model, calving ice is typically routed to the ocean as a solid ice flux.
In the absence of an explicit iceberg model, the ocean model supplies the heat needed to melt the ice.
The ocean cools and freshens accordingly.
