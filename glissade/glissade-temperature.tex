
\section{Temperature Solver}
\label{sc:glissade-temperature}

As discussed in Section~\ref{sc:glide_temp_solver}, the thermal evolution of the ice sheet is given by

\begin{equation}
  \label{gliss.eq.temp_evol}
  \frac{\partial T}{\partial t} = 
  \frac{k}{\rho c} {{\nabla }^{2}}T - \mathbf{u} \cdot \nabla T - w\frac{\partial T}{\partial z} + \frac{\Phi }{\rho c},
\end{equation}

\noindent
where $T$ is the temperature in $^\circ$C, $k$ is the thermal conductivity of ice, $c$ is the specific heat of ice, 
$\rho$ is the density, and $\Phi$ is the rate of heating due to internal deformation and dissipation.
This equation describes the conservation of internal energy under horizontal and vertical 
diffusion (the first term on the RHS), horizontal and vertical advection 
(the second and third terms, respectively), and internal heat dissipation (the last term).
Glide solves this equation in module {\tt glide\_temp.F90}.
Glissade takes a different approach, dividing the temperature evolution into an advective term
and a diffusion/dissipation term.
Module {\tt glissade\_temp.F90} solves for diffusion and internal dissipation:

\begin{equation}
  \label{gliss.eq.vert_temp_evol}
  \frac{\partial T}{\partial t} = \frac{k}{\rho c}{{\nabla }^{2}}T + \frac{\Phi }{\rho c},
\end{equation}

\noindent
as described in this section.
The advective part of \eqref{gliss.eq.temp_evol} is described
in Section \ref{sc:glissade-transport}.

Glissade's vertical discretization of temperature is also different from that of Glide.
In Glide, temperature is located at each of $nz$ vertical levels.
In Glissade, internal temperatures are located at the midpoints of the $nz-1$ layers.
Temperature is also defined at the upper and lower ice surface, giving a total
of $nz+1$ temperature points in each column.  The upper surface temperature is denoted
by $T_0$ and the lower surface temperature by $T_{nz}.$

This convention makes it fairly straightforward
to advect temperature conservatively.  The total internal energy in a column 
is the sum over layers of $\rho c T \Delta z$, where $\Delta z$ is the layer thickness.
This internal energy is conserved under transport (see Section \ref{sc:glissade-transport}).
$T_0$ and $T_{nz}$, which are determined by the boundary conditions, 
are associated with infinitesimally thin layers that do not contain
any internal energy.

The following sections describe how the terms in \eqref{gliss.eq.vert_temp_evol} are computed,
how the boundary conditions are specified, and how the equation is solved. 

\subsection{Vertical diffusion}

Computing the vertical diffusion term requires a discretization for ${\nabla}^{2}T$.
As in Glide (Section~\ref{sc:glide_vert_diff}), we assume that horizontal diffusion is
negligible compared to vertical diffusion:

\begin{equation}
  {{\nabla }^{2}}T\simeq \frac{{{\partial }^{2}}T}{\partial {{z}^{2}}}=\frac{1}{{{H}^{2}}}\frac{{{\partial }^{2}}T}{\partial {{\sigma }^{2}}},
\end{equation}
%
where the last equality follows from $\sigma = (s-z)/H$.

In $\sigma$--coordinates, the central difference formulas
for first derivatives at the upper and lower interfaces of layer $k$ are

\begin{equation}
  \label{gliss.eq.temp_d1}
  \begin{split}
    {{\left. \frac{\partial T}{\partial \sigma } \right|}_{{{\sigma }_{k}}}} =
    \frac{{{T}_{k}}-{{T}_{k-1}}}{{{{\tilde{\sigma }}}_{k}}-{{{\tilde{\sigma }}}_{k-1}}},\\
    {{\left. \frac{\partial T}{\partial \sigma } \right|}_{{{\sigma }_{k+1}}}} =
    \frac{{{T}_{k+1}}-{{T}_{k}}}{{{{\tilde{\sigma }}}_{k+1}}-{{{\tilde{\sigma }}}_{k}}},
  \end{split}
\end{equation}
%
where $\tilde{\sigma}_k$ is the value of $\sigma$ at the midpoint of layer $k$, 
halfway between $\sigma_{k}$ and $\sigma_{k+1}$:

\begin{equation}
  \tilde{\sigma}_k = \frac{\sigma_{k+1} - \sigma_{k}} {2}.
\end{equation}
%
The second partial derivative, defined at the midpoint of layer $k$, is
\begin{equation}
  \label{gliss.eq.temp_d2}
        {{\left. \frac{{{\partial }^{2}}T}{\partial {{\sigma }^{2}}} \right|}_{{{{\tilde{\sigma }}}_{k}}}} =
        \frac{{{\left. \frac{\partial T}{\partial \sigma } \right|}_{{{\sigma }_{k+1}}}} - {{\left. \frac{\partial T}{\partial \sigma } \right|}_{{{\sigma }_{k}}}}} 
             {{{\sigma }_{k+1}}-{{\sigma }_{k}}}
\end{equation}
%
Inserting \eqref{gliss.eq.temp_d1} in \eqref{gliss.eq.temp_d2}, we obtain
the required vertical diffusion term:

\begin{multline}
    \label{gliss.eq.temp_d3}
          {{\left. \frac{{{\partial }^{2}}T}{\partial {{\sigma }^{2}}} \right|}_{{{{\tilde{\sigma }}}_{k}}}} =
          \frac{{{T}_{k-1}}}{\left( {{{\tilde{\sigma }}}_{k}}-{{{\tilde{\sigma }}}_{k-1}} \right)\left( {{\sigma }_{k+1}}-{{\sigma }_{k}} \right)}
          - {{T}_{k}}\left( \frac{1}{\left( {{{\tilde{\sigma }}}_{k}}-{{{\tilde{\sigma }}}_{k-1}} \right)\left( {{\sigma }_{k+1}}-{{\sigma }_{k}} \right)}+\frac{1}{\left( {{{\tilde{\sigma }}}_{k+1}}-{{{\tilde{\sigma }}}_{k}} \right)\left( {{\sigma }_{k+1}}-{{\sigma }_{k}} \right)} \right)\\
          +\frac{{{T}_{k+1}}}{\left( {{{\tilde{\sigma }}}_{k+1}}-{{{\tilde{\sigma }}}_{k}} \right)\left( {{\sigma }_{k+1}}-{{\sigma }_{k}} \right)}.
\end{multline}
%
\subsection{Heat dissipation}

In higher-order models the internal heating rate $\Phi$ in \eqref{gliss.eq.vert_temp_evol} 
is given by the tensor product of strain rate and stress:

\begin{equation}
  \label{gliss.eq.dissipation1}
  \Phi ={\dot{{\varepsilon} }_{ij}}{{\tau }_{ij}}.
\end{equation}

\noindent
The effective strain rate and effective stress (cf. \eqref{gliss.eq.effective_strain_rate})
are defined by
\begin{equation}
  \label{gliss.eq.eff_stress_strain}
   \dot{{\varepsilon}}_{e}^{2}=\frac{1}{2}{\dot{{\varepsilon} }_{ij}}{\dot{{\varepsilon} }_{ij}}, \quad {\tau}_{e}^{2}=\frac{1}{2} {{\tau }_{ij}}{{\tau }_{ij}}.
\end{equation}

\noindent
It follows from \eqref{gliss.eq.dissipation1} and \eqref{gliss.eq.eff_stress_strain} that
\begin{equation}
  \label{gliss.eq.dissipation2}
  \Phi = 2 {\dot{{\varepsilon} }_e}{{\tau }_e}.
\end{equation}

\noindent
Eq. \eqref{gliss.eq.L1L2_eta2}, which defines the effective viscosity, implies

\begin{equation}
  \dot{\varepsilon}_e = \frac{\tau_e}{2 \eta}, 
\end{equation}

\noindent
which can be substituted in \eqref{gliss.eq.dissipation2} to give
\begin{equation}
  \label{gliss.eq.dissipation3}
  \Phi = \frac{\tau_e^2}{\eta}.
\end{equation}
%
Both terms on the RHS of \eqref{gliss.eq.dissipation3} are available to the temperature solver,
since the higher-order velocity solver computes $\eta$ during matrix assembly
and diagnoses $\tau_e$ from $\eta$ and $\dot{\varepsilon}_{ij}$ at the end of the calculation.

\subsection{Boundary conditions}
The temperature $T_0$ at the upper boundary is set to the surface air temperature $T_{\mathrm{air}}$.
(In coupled climate applications, $T_0$ may be passed in by the climate model.)
The diffusive heat flux at the upper boundary (defined as positive up) is

\begin{equation}
  F_d^{\mathrm{top}} = \frac{k}{H} \frac{T_1 - T_0}{\tilde{\sigma}_1}.
\end{equation}

\noindent
(The denominator contains just one term because $\sigma_0 = 0$.)
Optionally, this flux can be returned to the climate model as a lower boundary
condition in the land-surface model.

The lower ice boundary is more complex. For grounded ice there are three heat sources and sinks.
First, the diffusive flux from the bottom surface to the ice interior 
(positive up) is

\begin{equation}
  F_d^{\mathrm{bot}} = \frac{k}{H} \frac{T_{nz} - T_{nz-1}}{1 - \tilde{\sigma}_{nz-1}}.
\end{equation}

\noindent
Second, there is a geothermal heat flux $F_g$ to the lower boundary. This is typically 
prescribed as a constant ($\sim 0.05$ W m$^{-2}$) or read from an input file.
Finally, there is a frictional heat flux associated with basal sliding,
given by \citep[p. 418]{Cuffey2010}

\begin{equation}
  F_f = \mathbf{\tau_b} \cdot \mathbf{u_b},
\end{equation}

\noindent
where $\mathbf{\tau_b}$ and $\mathbf{u_b}$ are 2D vectors of basal shear stress and basal velocity,
respectively. With a friction law of the form \eqref{gliss.eq.beta_tau}, this becomes

\begin{equation}
  F_f = \beta \sqrt{u_b^2 + v_b^2}.
\end{equation}

If the basal temperature $T_{nz} < T_{\mathrm{pmp}}$ 
(where $T_{\mathrm{pmp}} = T + 8.7\cdot 10^{-4}(s-z)$ is the pressure melting point temperature),
then the fluxes at the lower boundary must balance:

\begin{equation}
  F_g + F_f = F_{d}^{\mathrm{bot}}.
\end{equation}

\noindent
In other words, the energy supplied by geothermal heating and sliding friction is equal
to the energy removed by vertical diffusion.
If, on the other hand, $T_{nz} = T_{\mathrm{pmp}}$, then the
lower surface temperature is fixed and the net flux is used to melt or freeze ice at the boundary:

\begin{equation}
  \label{gliss.eq.basal_melt}
  M_b = \frac{F_g + F_f - F_d^{\mathrm{bot}}}{\rho L},
\end{equation}
 
\noindent
where $M_b$ is the melt rate and $L$ is the latent heat of melting.
Melting generates basal water, which may either stay in place or flow downstream
(possibly replaced by water from upstream), depending on the
parameterization chosen in the config file (\texttt{basal\_water}; see \ref{ug.sec.config}).
We hold $T_{nz} = T_{\mathrm{pmp}}$ as long as basal water is present.

For floating ice the basal boundary condition is simpler; $T_b$ is simply
set to the freezing temperature $T_f$ of seawater. Optionally, a melt rate could
be prescribed at the lower surface.

\subsection{Vertical temperature solution}

Eq. \eqref{gliss.eq.vert_temp_evol} can be discretized for an ice layer $k$ as

\begin{equation}
  \label{gliss.eq.dTdt1}
  \frac{T_{k}^{n+1}-T_{k}^{n}}{\Delta t} =
  \frac{k}{\rho c}\left[ {{a}_{k}}T_{k-1}^{n+1}-({{a}_{k}}+{{b}_{k}})T_{k}^{n+1}+{{b}_{k}}T_{k+1}^{n+1} \right]+\frac{{{\Phi }_{k}}}{\rho c},
\end{equation}

\noindent
where the coefficients $a_k$ and $b_k$ are given by \eqref{gliss.eq.temp_d3}, $n$ is the current time level,
and $n+1$ is the new time level.  The vertical diffusion terms are evaluated at the new time level,
making the discretization backward Euler (i.e., fully implicit) in time.  A Crank--Nicolson formulation, in which the temperature
terms are evaluated at time $n+1/2$, is also available.  Although Crank--Nicolson is second-order-accurate
in time (compared to first-order for backward Euler), it can lead to temperature oscillations in thin ice.
For this reason, backward Euler is the default.

Eq. \eqref{gliss.eq.dTdt1} can be rewritten as
\begin{equation}
  \label{gliss.eq.dTdt2}
  -{{\alpha }_{k}}T_{k-1}^{n+1} + (1+{{\alpha }_{k}}+{{\beta }_{k}})T_{k}^{n+1} - {{\beta }_{k}}T_{k+1}^{n+1} =
  T_{k}^{n} + \frac{{{\Phi }_{k}}\Delta t}{\rho c},
\end{equation}

\noindent
where

\begin{equation}
   {{\alpha }_{k}}=\frac{{{a}_{k}}\Delta t}{\rho c}, \quad {{\beta }_{k}}=\frac{{{b}_{k}}\Delta t}{\rho c}.
\end{equation}

\noindent
At the upper surface, $T_0 = T_{\mathrm{air}}$.  At the lower surface we have either 
a temperature boundary condition ($T_{nz} = T_{\mathrm{pmp}}$ for grounded ice, or
$T_{nz} = T_f$ for floating ice) or a flux boundary condition:

\begin{equation}
  \label{gliss.eq.lower_flux_bc}
  {{F}_{f}}+{{F}_{g}}-\frac{k}{H}\frac{T_{nz}^{n+1}-T_{nz-1}^{n+1}}{1-{{{\tilde{\sigma }}}_{nz-1}}} = 0,
\end{equation}

\noindent
which can be rearranged to give

\begin{equation}
  \label{gliss.eq.lower_flux_bc2}
  -T_{nz-1}^{n+1}+T_{nz}^{n+1} = \frac{\left( {{F}_{f}}+{{F}_{g}} \right)H\left( 1-{{{\tilde{\sigma }}}_{nz-1}} \right)}{k}.
\end{equation}

\noindent
In each ice column the above equations form a tridiagonal system that is easily solved for $T_k$.

Occasionally, the solution $T_k$ in one or more layers will exceed $T_{\mathrm{pmp}}$ for the layer.
If so, we set $T_k = T_{\mathrm{pmp}}$ and use the extra energy to melt ice internally.
This melt is assumed (not very realistically) to drain immediately to the bed.

If \eqref{gliss.eq.basal_melt} applies, we compute $M_b$ and adjust the basal water depth.
When the basal water goes to zero, $T_{nz}$ is set to a value slightly below $T_{\mathrm{pmp}}$
so that the flux boundary condition will apply during the next time step.

Given the new 3D temperature field, we can compute the flow factors $A(T)$ required for
the velocity solution.  We assume an Arrhenius relationship:

\begin{equation}
  \label{gliss.eq.arrhenius}
  A(T_{\mathrm{pmp}})=a e^{-Q/RT_{\mathrm{pmp}}},
\end{equation}

\noindent
where $a$ is a temperature--independent material constant (given by \citet{PatersonBudd:1982}),
$Q$ is the activation energy for creep and $R$ is the universal gas constant.

\subsection{Enthalpy model}

An alternative vertical temperature solver based on enthalpy is under development.
In this scheme, water that melts internally can be retained in the ice
instead of draining instantly to the bed. The retained water reduces the ice
viscosity compared to that of pure ice. Although CISM includes source code
for an enthalpy model, the model is not yet scientifically supported.


