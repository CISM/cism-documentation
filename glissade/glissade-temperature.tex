
\section{Temperature Solver}
\label{sc:glissade-temperature}

As discussed in Section (insert ref here), the thermal evolution of the ice sheet is given by

\begin{equation}
  \label{gliss.eq.temp_evol}
  \frac{\partial T}{\partial t} = 
  \frac{k}{\rho c} {{\nabla }^{2}}T - \bf{u}\cdot \nabla T - w\frac{\partial T}{\partial z} + \frac{\Phi }{\rho c}
\end{equation}

\noindent
where $T$ is the temperature in $^\circ$C, $k$ is the thermal conductivity of ice, $c$ is the specific heat of ice, 
$\rho$ is the density, and $\Phi$ is the rate of heating due to internal deformation and dissipation.
This equation describes the conservation of internal energy under horizontal and vertical 
diffusion (the first term), horizontal and vertical advection (the second and third terms), and 
internal heat dissipation (the last term).
Glide solves this equation in module {\tt glide\_temp.F90}.
Glissade takes a different approach, dividing the temperature evolution into an advective term
and a diffusion/dissipation term.
Module {\tt glissade\_temp.F90} solves for diffusion and internal dissipation:

\begin{equation}
  \label{gliss.eq.vert_temp_evol}
  \frac{\partial T}{\partial t} = \frac{k}{\rho c}{{\nabla }^{2}}T + \frac{\Phi }{\rho c},
\end{equation}

\noindent
as described in this section.
The advective part of \eqref{gliss.eq.temp_evol} is described separately
in Section \ref{sc:glissade-transport}.

Glissade's vertical discretization of temperature is also different from that of Glide.
In Glide, temperature is located at each of $nz$ vertical levels.
In Glissade, however, internal temperatures are located at the midpoints of the $nz-1$ layers.
Temperature is also defined at the upper and lower ice surface, giving a total
of $nz+1$ temperature points in each column.  This convention makes it more straightforward
to advect temperature conservatively.  The total internal energy is the sum over layers of
$\rho c T \Delta z$, where $\Delta z$ is that layer thickness.  
The upper and lower surface temperatures, which are determined by the boundary conditions, 
are assumed to belong to infinitesimally thin layers, so they are not associated with
any internal energy.

The following sections describe how the terms in \eqref{gliss.eq.vert_temp_evol} are computed,
how the boundary conditions are determined, and how the equation is solved. 

\subsection{Vertical diffusion}

Computing the vertical diffusion term requires a discretization for ${\nabla}^{2}T$.
As in Glide (Section~\ref{sc:glide_vert_diff}), we assume that horizontal diffusion is
negligible compared to vertical diffusion:

\begin{equation}
  {{\nabla }^{2}}T\simeq \frac{{{\partial }^{2}}T}{\partial {{z}^{2}}}=\frac{1}{{{H}^{2}}}\frac{{{\partial }^{2}}T}{\partial {{\sigma }^{2}}}
\end{equation}
%
where the last equality follows from $\sigma = (s-z)/H$.

In $\sigma$--coordinates, the central difference formulas
for first derivatives at the upper and lower interfaces of layer $k$ are

\begin{equation}
  \label{gliss.eq.temp_d1}
  \begin{split}
    {{\left. \frac{\partial T}{\partial \sigma } \right|}_{{{\sigma }_{k}}}} =
    \frac{{{T}_{k}}-{{T}_{k-1}}}{{{{\tilde{\sigma }}}_{k}}-{{{\tilde{\sigma }}}_{k-1}}},\\
    {{\left. \frac{\partial T}{\partial \sigma } \right|}_{{{\sigma }_{k+1}}}} =
    \frac{{{T}_{k+1}}-{{T}_{k}}}{{{{\tilde{\sigma }}}_{k+1}}-{{{\tilde{\sigma }}}_{k}}}.
  \end{split}
\end{equation}
%
Here, $\tilde{\sigma}_k$ is the value of $\sigma$ at the midpoint of layer $k$, 
halfway between $\sigma_{k}$ and $\sigma_{k+1}$:

\begin{equation}
  \tilde{\sigma}_k = \frac{\sigma_{k+1} - \sigma_{k}} {2}.
\end{equation}
%
The second partial derivative is
\begin{equation}
  \label{gliss.eq.temp_d2}
        {{\left. \frac{{{\partial }^{2}}T}{\partial {{\sigma }^{2}}} \right|}_{{{{\tilde{\sigma }}}_{k}}}} =
        \frac{{{\left. \frac{\partial T}{\partial \sigma } \right|}_{{{\sigma }_{k+1}}}} - {{\left. \frac{\partial T}{\partial \sigma } \right|}_{{{\sigma }_{k}}}}} 
             {{{\sigma }_{k+1}}-{{\sigma }_{k}}}
\end{equation}
%
Inserting \eqref{gliss.eq.temp_d1} into \eqref{gliss.eq.temp_d2}, we obtain
the required vertical diffusion term:

\begin{multline}
    \label{gliss.eq.temp_d3}
          {{\left. \frac{{{\partial }^{2}}T}{\partial {{\sigma }^{2}}} \right|}_{{{{\tilde{\sigma }}}_{k}}}} =
          \frac{{{T}_{k-1}}}{\left( {{{\tilde{\sigma }}}_{k}}-{{{\tilde{\sigma }}}_{k-1}} \right)\left( {{\sigma }_{k+1}}-{{\sigma }_{k}} \right)}
          - {{T}_{k}}\left( \frac{1}{\left( {{{\tilde{\sigma }}}_{k}}-{{{\tilde{\sigma }}}_{k-1}} \right)\left( {{\sigma }_{k+1}}-{{\sigma }_{k}} \right)}+\frac{1}{\left( {{{\tilde{\sigma }}}_{k+1}}-{{{\tilde{\sigma }}}_{k}} \right)\left( {{\sigma }_{k+1}}-{{\sigma }_{k}} \right)} \right)\\
          +\frac{{{T}_{k+1}}}{\left( {{{\tilde{\sigma }}}_{k+1}}-{{{\tilde{\sigma }}}_{k}} \right)\left( {{\sigma }_{k+1}}-{{\sigma }_{k}} \right)}.
\end{multline}
%
\subsection{Heat dissipation}

In higher-order models, the internal heating $\Phi$ in \eqref{gliss.eq.vert_temp_evol} 
is given by the tensor product of strain rate and stress:

\begin{equation}
  \label{gliss.eq.dissipation1}
  \Phi ={\dot{{\varepsilon} }_{ij}}{{\tau }_{ij}}.
\end{equation}

\noindent
The effective strain rate and effective stress (cf. \eqref{gliss.eq.effective_strain_rate})
are defined by
\begin{equation}
  \label{gliss.eq.eff_stress_strain}
   \dot{{\varepsilon}}_{e}^{2}=\frac{1}{2}{\dot{{\varepsilon} }_{ij}}{\dot{{\varepsilon} }_{ij}}, \quad {\tau}_{e}^{2}=\frac{1}{2} {{\tau }_{ij}}{{\tau }_{ij}}.
\end{equation}

\noindent
It follows from \eqref{gliss.eq.dissipation1} and \eqref{gliss.eq.eff_stress_strain} that
\begin{equation}
  \label{gliss.eq.dissipation2}
  \Phi = 2 {\dot{{\varepsilon} }_e}{{\tau }_e}.
\end{equation}

\noindent
From \eqref{gliss.eq.L1L2_eta2} we can show that

\begin{equation}
  \dot{\varepsilon}_e = \frac{\tau_e}{2 \eta}, 
\end{equation}

\noindent
which can be substituted in \eqref{gliss.eq.dissipation2} to obtain
\begin{equation}
  \label{gliss.eq.dissipation3}
  \Phi = \frac{\tau_e^2}{\eta}.
\end{equation}
%
Both terms on the RHS are available to the temperature solver, since the 
higher-order velocity solver computes $\eta$ during matrix assembly
and diagnoses $\tau_e$ from the strain rates at the end of the calculation.
