\section{SOFTWARE INTEGRATION \label{ch:softintegration}}

%%%%%%%%%%%%%%%%%%%%%%%%%%%%%%%%%%%%%%%%%%%%%%%%%%%%%%%%%%%%%%%%%%%%%%%%%%%%%%%%
\subsection{Overview}\label{sec:ch4over}

The architecture combining GLIMMER-CISM with PHAML can be partitioned into a few key areas:  grid mapping, callback definitions, data manipulation between master and slaves, build systems, debugging techniques, problem definition, and the integration of results into the ice sheet dynamics.  Some of these areas are problems beyond the scope of this thesis, but are still worth noting.  A large portion of the issues though have to do with the code between the two components and making sure they're working together properly.  The resolutions to these issues are discussed in this chapter.

\subsubsection{Files}

The majority of the code to link the two software components lies in three files: 

\begin{itemize}
\item \href{http://svn.berlios.de/svnroot/repos/glimmer-cism/glimmer-cism2/libphaml/phaml\_support.F90}{phaml\_support.F90} - The support file contains the code which does mesh generation, boundary marker assignment, and some helper functions for generating the mesh.
\item \href{http://svn.berlios.de/svnroot/repos/glimmer-cism/glimmer-cism2/libphaml/phaml\_pde.F90}{phaml\_pde.F90} - This file acts as a broker between the different problem specific PHAML modules.  This becomes the slave process which PHAML will call, and it determines which CISM PHAML module needs to be called.
\item \href{http://svn.berlios.de/svnroot/repos/glimmer-cism/glimmer-cism2/libphaml/phaml\_user\_mod.F90}{phaml\_user\_mod.F90} - The user module contains data and functions that are needed across all the PHAML callbacks.  The implementation and purpose is discussed further in section \ref{sec:ch4usermod}.
\end{itemize}

An example was also created to demonstrate how the PHAML components and support functionality can be used as a module in GLIMMER-CISM.  A new module requires two files to be created.  The example modules that demonstrates this are located in the following files:

\begin{itemize}
\item \href{http://svn.berlios.de/svnroot/repos/glimmer-cism/glimmer-cism2/libphaml/phaml\_example.F90}{phaml\_example.F90} - Contains PHAML calls with tweaks specific for the problem and any other data or functions needed.
\item \href{http://svn.berlios.de/svnroot/repos/glimmer-cism/glimmer-cism2/libphaml/phaml\_example\_pde.F90}{phaml\_example\_pde.F90} - This file has all the problem specific PDE callback definitions that will be invoked via phaml\_pde.F90.
\end{itemize}

The example modules must still be called from within GLIMMER-CISM, so a very basic driver is also included which shows how the example modules are used and how the data from GLIMMER-CISM is passed to the modules and vice-versa.

\begin{itemize}
\item \href{http://svn.berlios.de/svnroot/repos/glimmer-cism/glimmer-cism2/libphaml/simple\_phaml.F90}{simple\_phaml.F90} - A simple driver demonstrating how to use the example modules and data structures.
\end{itemize}

In order to work with the GLIMMER-CISM data, PHAML also needs datasets within the standard ice-sheet model data.  Thus another custom datatype in CISM has been added to the model called `phaml' that is used to hold any data PHAML may need during a simulation or any data CISM might want to store from PHAML-this might include initial conditions, flags, boundary conditions, etc.  A listing of the datatypes and variable names in this custom type is located in appendix \ref{ch:phamlvars}.

\subsubsection{Master/Slaves}

PHAML uses subprocesses in order to parallelize the work being done over processors or machines.  Because of these `slaves' PHAML uses MPI in order to communicate back and forth with them.  This can cause issues if the PHAML master process does not close correctly.  Whenever a PHAML module is used these slave processes will be launched and for a long simulation it is necessary to check to make sure there are no orphaned phaml\_slave processes running because they will eat up CPU time and eventually cause some issues with the system.



\subsubsection{Function Overview}

Following the program can be problematic given the master and slave processes as well as the callbacks.  Figure \ref{fig:func} shows the basic function call diagram for the overall process beginning with a driver in CISM loading a phaml\_module.  The colors represent different modules and the dashed lines represent a call made (either direction) between the master and slave.  The assumption is also made that functions return data, but this diagram is merely showing the subroutine calls and not data flow.

The chart is read left to right and top to bottom meaning the calls on the first level at the left are executed before the calls at the same level to the right.  At each function the subsequent function calls below it are then executed.

The update\_usermod subroutine is called multiple times which is explained further in section \ref{sec:ch4usermod}.  The call `master\_to\_slaves' is a subroutine within the master process that the slave calls.  The master then returns the usermod data which the master has.  In this way whenever the usermod changes in the main process, the changes are propagated down to the slaves.



\addfigurepdf{func}{1.0}{Function Call Diagram: An overview of the function calls and how the module functions are connected}{An overview of the function calls and how the module functions are connected.  Dashed lines indicate a call between master and slave.}{fig:func}
  



%%%%%%%%%%%%%%%%%%%%%%%%%%%%%%%%%%%%%%%%%%%%%%%%%%%%%%%%%%%%%%%%%%%%%%%%%%%%%%%%
\newpage

\subsection{PHAML PDE}\label{sec:ch4phamlpde}

The callbacks that PHAML uses are declared in the phaml\_pde module.  Since the subroutines are problem specific and therefore must be tied to the purpose of the GLIMMER-CISM/PHAML module, the functions defined in phaml\_pde act as a broker to find the correct callback for a given module.  The names of the callbacks can not be changed because PHAML looks for them explicitly which is why this broker model is necessary.  The functions can pass the data to the correct module function by looking at the modnum (module number) declared in the user module

Another important point about the GLIMMER-CISM modules is that since these modules have a specific number associate with them and there is only one user module, only one CISM PHAML module can be in scope at a time.  This means that for many computational tasks the CISM PHAML modules will have to be called, run, the results stored, and then discarded until later needed.


%%%%%%%%%%%%%%%%%%%%%%%%%%%%%%%%%%%%%%%%%%%%%%%%%%%%%%%%%%%%%%%%%%%%%%%%%%%%%%%%
\subsection{PHAML Support}\label{sec:ch4supp} 

The PHAML support module contains the mesh generation code that is problem independent and can be used by all PHAML modules that might be created for different applications.  The functions are all support functions of the mesh creation in one facet or another.  The individual functions are described in Appendix \ref{ch:libphamlfunc} section \ref{sec:libfuncsupp}.


\subsubsection{Mesh Generation}

Since PHAML works by solving a partial differential equation on a specific domain, the full GLIMMER-CISM grid (a 2-dimensional matrix) can not be used since part of that domain defines the boundaries and the area around the ice-sheet.  This means the grid must be interpreted and a mesh has to be output which defines only the parts of data relevant to the ice-sheet itself.  This procedure is detailed more thoroughly in Chapter \ref{ch:modelinput} section \ref{sec:chp3map}.  The code is contained within the support functionality since it doesn't change.  

The mesh generation will likely happen multiple times per simulation since upon each time step the glacier is likely to change.  This means each time the glacier changes the mesh must be created anew and the simulation will continue with the new mesh of the ice-sheet.


%%%%%%%%%%%%%%%%%%%%%%%%%%%%%%%%%%%%%%%%%%%%%%%%%%%%%%%%%%%%%%%%%%%%%%%%%%%%%%%%
\subsection{PHAML Usermod}\label{sec:ch4usermod}

A very important feature of PHAML that must be handled carefully is the idea of the usermod (user module) that must be declared and maintained.  The usermod module is the definition of all variables that might be needed in any of the PHAML callbacks or internal functions.  These variables must also be manually passed to the slaves in order for the callbacks to work correctly.  Therefore, any callback that might need data from GLIMMER-CISM (initial conditions, boundary conditions, PDE coefficients) must get that information from the user module.  

In order to send the information via usermod the data must be set in phaml\_user\_mod.  This process can be tricky when dealing with dynamic arrays so care must be taken.  PHAML can only send data to the slaves by passing an integer array and a real array that is one-dimensional.  In the case of GLIMMER-CISM 2-dimensional arrays represent the model and are needed in order to set the conditions correctly.  Since the arrays are dynamic the master must first send all the size information to the slaves, then call update\_usermod again to send the array data.  As mentioned the array data must first be reshaped and combined if more than one array is needed, and then passed to the slaves where it is split and then reshaped back to its original form.  These transformations must be written specifically for the needed data which is why the user module includes data manipulation functions to ease this transition.  These functions are outlined in appendix \ref{ch:libphamlfunc} section \ref{sec:libfuncuser}.  There may be occasions where problem specific data is needed which is why the update\_usermod subroutine actually lies within the phaml\_example\_pde module, so that it can be tweaked when needed.  The general process is shown in figure \ref{fig:flow}.


\addfigurepdf{flow}{0.53}{Update Usermod Flowchart: The update\_usermod flow of instructions to format and pass data to the slave processes}{The update\_usermod flow of instructions to format and pass data to the slave processes.}{fig:flow}
  

%%%%%%%%%%%%%%%%%%%%%%%%%%%%%%%%%%%%%%%%%%%%%%%%%%%%%%%%%%%%%%%%%%%%%%%%%%%%%%%%
\subsection{PHAML Example Module}\label{sec:ch4example}

The goal of the libphaml addition for GLIMMER-CISM is to provide an interface for PHAML as well as being able to make problem specific drivers.  A version might be created to calculate temperature, flux, thickness, or any other number of fields, but each of these problems will model a different PDE and need different data sets in order to be solved.  Thus, the name of this file is phaml\_example because it is not intended to be a main driver, but merely to illustrate how one is written, what components and functions are needed, and how to use it.


\subsubsection{PHAML Example}

The file phaml\_example.F90 defines the interface for the GLIMMER-CISM code to use the PHAML library.  The method in which they're called and the options for the PHAML functions can all be modified if needed based on the purpose of the module.  The basic usage is shown in the example module as well as the necessary calls for setting up, maintaining, and closing a PHAML solution, the user module, and all other necessary components.  How this module is used in CISM is shown in appendix \ref{ch:usage}

\subsubsection{PHAML Example PDE}

The phaml\_example\_pde.F90 file contains all of the callbacks necessary to define the PDE being solved and how it acts at each step.  These subroutines define the coefficients, functions, and variables of the PDE \ref{eq:phamlmain} that PHAML is solving.  These functions are all called from the phaml\_pde module based on the modnum variable in the user module.  The functions themselves are described in section \ref{sec:libphamlcall} of the appendix and the user module functions and variables are explained in section \ref{sec:libfuncuser}. 

%%%%%%%%%%%%%%%%%%%%%%%%%%%%%%%%%%%%%%%%%%%%%%%%%%%%%%%%%%%%%%%%%%%%%%%%%%%%%%%%
\subsection{Build System}\label{sec:ch4make}

Incorporating one software system as a library of another can create a lot of issues.  One of the most daunting ones is the build system.  Combining them requires making sure all dependencies get included in the build, that they're built in the correct order, and that certain aspects remain optional in the build process.

Making sure that PHAML remains easy to include, but yet still separate is extremely important to this work.  For the majority of simulations needed to be run by GLIMMER-CISM, PHAML will not be needed which means it should not be required to be installed or built along with the project.  Having this convenience requires a bit of convolution in the build system.  GLIMMER-CISM uses the Autoconf/Make tools for handling the compiling process with all the possible options and dependencies.  

The amount of code needed to handle PHAML by itself is not excessive, but it takes quite a bit of time to understand exactly what different parts of the build system are doing.  PHAML also requires some extra building procedures since the PHAML specific modules must also be built as a slave process for the master.  This added complexity must be handled carefully, and also requires extra scripting in order to build and install the slave separate from the main CISM executables.

The extra dependencies that PHAML requires can be fairly minimal, but can also be quite expansive depending on the features used.  The most invasive libraries would be if the OpenGL extensions were desired for debugging or visualization preferences.  These are excellent options to have since they are very versatile and provide a unique perspective on how PHAML is working and what the solution is.  Including these however, requires a lot of dependency checking, additional compile tags, an additional graphics executable to be built, and can introduce a lot of possibility for build errors.


%%%%%%%%%%%%%%%%%%%%%%%%%%%%%%%%%%%%%%%%%%%%%%%%%%%%%%%%%%%%%%%%%%%%%%%%%%%%%%%%
\subsection{Documentation}\label{sec:ch4doc}

Documentation is important in any software project, but especially so when interfacing between projects.  The difficulty lies in the need to thoroughly understand both pieces of software in order to use them effectively.  This requires using the documentation from each project individually, which is why the intermediate documentation is quite necessary.  There must be explanation as to how the frameworks as explained in the individual software resources have been modified or mapped in order to work with other tools.

Another issue with documentation is formatting and standardization within the project.  GLIMMER-CISM uses Doxygen which is a documentation system that has many advantages:  configurable, can generate manual documentation in many formats, and it works with many languages.  \citep{doxygen:website} This allows all code added to GLIMMER-CISM to use code documentation as reference, so that code documentation and usage reference can be written only once.

The interface code for this project conforms to the Doxygen standard.  The example module does not since these PHAML functions may change greatly for different uses, and so only basic definitions are provided as provided by the PHAML documentation.
