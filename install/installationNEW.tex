
\section{Getting and Installing CISM}
\label{sec:getcode}

\href{http://oceans11.lanl.gov/cism/}{CISM}\footnote{\texttt{http://oceans11.lanl.gov/cism/}} 
is a relatively complex system of libraries and programs which build on other libraries. 
This section documents how to download and install CISM and its prerequisites.
Many common problems and questions can be addressed using the 
\href{http://forum.cgd.ucar.edu/forums/ice-sheet-modeling-cism}{user discussion boards}\footnote{\texttt{http://forum.cgd.ucar.edu/forums/ice-sheet-modeling-cism}}
. 
A CISM users mailing list is also available and can be subscribed to by sending an email
to \texttt{cism-users@googlegroups.com}\footnote{In order to subscribe from a non-Google email
address, you should first make sure to completely log out from any Google sites (e.g., Gmail) before sending 
your request. If you do not, it will automatically try to associate your request with your Gmail account instead.}.
Please report unresolved problems using the bug reporting facility at the 
\href{https://github.com/CISM/cism/issues}{CISM Github website}\footnote{\texttt{https://github.com/CISM/cism/issues}}. 

CISM is distributed as source code, and a reasonably complete build environment is therefore required to compile the model. 
For UNIX and LINUX based systems (including Mac), the \href{http://www.cmake.org/}{CMake} build system is used to build the model. 
Sample build scripts for a number of standard architectures are included, as are working build scripts 
for a number of large-scale, high-performance-computing architectures 
(e.g., \textit{Yellowstone} (CISL), \textit{Titan} (OLCF), and \textit{Hopper} (NERSC) ). 

There are two ways to get the source code:

\begin{enumerate}

\item \href{https://github.com/CISM/cism/releases}{Download}\footnote{\texttt{https://github.com/CISM/cism/releases}} a {\it released} version of the code as an archive (.zip or .tar.gz file).
\item Clone the code from the \href{https://github.com/CISM/cism}{CISM Github repository}\footnote{\texttt{https://github.com/CISM/cism}} using the following command: 
\texttt{git clone https://github.com/CISM/cism.git}
(It is also possible to clone the repository using the SSH protocol 
if you have an SSH keypair generated on your computer and attached to your GitHub account.  
See the \href{https://github.com/CISM/cism}{CISM Github repository webpage} and 
\href{https://help.github.com/articles/which-remote-url-should-i-use}{Github's help pages} for more information.)

\end{enumerate}

For beginners, downloading a zip archive of the latest release tag is recommended. 
More experienced users may want to download directly from the repository, 
as it will make updating the code easier in the future.

In either case, a Fortran90 compiler is required.  
Other software dependencies include the \href{http://www.unidata.ucar.edu/packages/netcdf/index.html}{netCDF} library 
(used for data I/O) and a \href{http://www.python.org}{Python} distribution 
(used to analyze dependencies and to automatically generate parts of the code) 
with a number of specific Python modules. Users who want to run the code in parallel will need to install MPI, 
and users who want access to the \textit{Trilinos} solver library will need to 
download and build \textit{Trilinos}, and link to it when building CISM. 
Finally, you will need CMake and Gnu Make to compile the code and link to the various third-party libraries. 

If you have not done so already, clone a tagged version of CISM or download an archive of the code, as noted above. Store the code or the unzipped/untarred archive in a location of your choice. More detailed build instructions, including instructions for the installation of supporting software, are given below.

% =====================================
% =====================================
\section{Installing Supporting Software for Basic (Serial) CISM}
% =====================================
% =====================================

Because the build process can be fairly complicated, we describe it in detail below, 
relying on the use of a package manager to handle many of the standard software dependencies. 
For each step we give specific instructions for both Mac OS X using MacPorts (in red boxes) and
Linux, specifically Ubuntu 12.10 (in blue boxes).  For different but related systems,
hopefully these instructions can be used as a guide.

CISM can be installed in either a serial or parallel configuration. The parallel mode
allows the model to be run on multiple processors which can greatly speed up execution.
This is a common configuration to use on supercomputing clusters, but can also be 
convenient on modern desktops and laptops which often have four or more cores available.
However, the parallel build requires additional supporting software, so we first 
detail how to build serial CISM.  For new users, it is recommended to first build
and successfully run serial CISM before moving on to the parallel build.  

\textbf{Note:} Glide, the shallow-ice dycore, can only run on a single processor, 
even when the code is built with full parallel support. This is also true of the SLAP
solver routines.

The instructions below assume the user has administrative privileges for
installing new software (note the extensive use of \texttt{sudo}).  
If you are working on a shared machine without
administrative privileges, you might proceed by assuming all needed packages are present and 
continue to the CISM installation section.  If you encounter problems, you can refer back to this
section to determine which packages might be missing or
problematic before contacting your system administrator.


\begin{mdframed}[style=mac] % V==============MAC================V
As mentioned above, we will take advantage of \href{http://www.macports.org/}{MacPorts}, 
a software package manager for Macs. This will allow us to install most of 
the base level software libraries needed by CISM with few complications. 

Go to \href{http://www.macports.org/install.php}{http://www.macports.org/install.php}, where you will find a range of ".pkg" installs available, including those for Mountain Lion, Lion, and Mavericks versions of Mac OS X. 

Installing MacPorts requires installing the Xcode developer toolset provided by Apple. Details of how to obtain Xcode vary by version of OS X. See MacPorts installation instructions and this \href{https://developer.apple.com/xcode/downloads/}{link} for details. Once Xcode is installed, you may need to additionally download the ``command line tool'' from the Preferences / Downloads menu of Xcode. 

Depending on computer security settings at your institution (firewalls, etc.), 
you may need to add proxy information so that Macports can communicate and 
download software from the outside world. All Macports software will be installed 
under \texttt{/opt/local/} by default. To add proxy information, after installing 
Macports, edit the configuration file at \texttt{/opt/local/etc/macports/macports.conf}.
(Note this is probably a read-only file that requires superuser permission to edit,
so you will need to edit the file with something like: \texttt{sudo vim /opt/local/etc/macports/macports.conf}). 
By searching for the text string ``proxy'', you will find the lines like 
\texttt{proxy\_http hostname:12345} near the bottom of the file. Enter your proxy 
information here as appropriate (e.g., \texttt{hostname:your\_host\_info\_here}).

If you have previously installed Macports but not updated it recently, it's generally a good idea to do so. Ideally, this should be done with admin or root privileges (you will be prompted to enter your password) using:

\texttt{sudo port selfupdate}

You will then be prompted to update any installed ports that are outdated, which you can do using:

\texttt{sudo port upgrade outdated}

To search for available software in Macports, type: 

\texttt{port search software-name}

Software is installed through Macports using the command:

\texttt{sudo port install software-name}

Additional Macports tips will follow inline below. Extensive documentation for Macports 
can be found at the \href{http://guide.macports.org}{Macports} website.
\end{mdframed}              % ^==============MAC================^


\begin{mdframed}[style=ubuntu] % V==============UBUNTU==========V
This Ubuntu instructions describe setting up supporting software and CISM in a Linux environment.
These instructions were written using a fresh installation of Ubuntu 12.10 but 
steps should be very similar in other versions of Ubuntu or other distributions of Linux.
Instructions make use of the command line tool for installing packages that comes with Ubuntu, 
\texttt{apt-get}.  Other package management tools (e.g., Software Center)
could also be used.

It's generally a good idea to synchronize your local package index files before
installing new software using \texttt{apt-get}:

\texttt{sudo apt-get update}

To search for available packages, type:

\texttt{apt-cache search software-name}

And to see detailed information about a package, type:

\texttt{apt-cache show software-name}

Packages are installed through apt-get using the command:

\texttt{sudo apt-get install software-name}

Additional apt-get tips will follow inline below. Extensive documentation for apt-get 
can be found at the \href{https://help.ubuntu.com/community/AptGet/Howto}{Ubuntu} website
and through man pages (\texttt{man apt-get}).
\end{mdframed}                 % ^==============UBUNTU==========^



% =====================================
\subsection{Install git version control software}
% =====================================
If you intend to download the CISM code as a git repository, you will need the \texttt{git} package installed.
If you prefer to download a zipped archive of the code, this step can be skipped.


\begin{mdframed}[style=mac] % V==============MAC================V
Install git with:

\texttt{sudo port install git}
\end{mdframed}              % ^==============MAC================^


\begin{mdframed}[style=ubuntu] % V==============UBUNTU==========V
Install git with:

\texttt{sudo apt-get install git}
\end{mdframed}                 % ^==============UBUNTU==========^


% =====================================
\subsection{Install the GCC compiler suite}
% =====================================

The GCC compiler suite contains compilers for C, C++, and, optionally, Fortran.
Fortran and C compilers are required for serial CISM, and a C++ compiler is also
needed for parallel CISM.  CISM is known to work with GNU gfortran compilers, 
Intel ifort, and PGI.  In these instructions we will use GNU compilers because they
have been extensively tested with CISM and are freely available.  Advanced users
are welcome to use other compilers of their choosing.

CISM has been tested extensively with \texttt{gfortran} versions 4.5 and 4.6.
Newer (or older) versions may also work, although version 4.8 introduces 
new features that may uncover issues.

\begin{mdframed}[style=mac] % V==============MAC================V
Searching for gcc with \texttt{port search gcc} will return:

\begin{verbatim}
gcc44 @4.4.7 (lang) 
    The GNU compiler collection 
...
\end{verbatim}

in addition to a lot of other information on available Macports installs related to the GCC (Gnu) compiler suite. 

Where possible, we want to make sure that all other software we build and install 
with Macports uses the version of GCC we choose to install. To date, we've had success 
with GCC 4.6.3 (others may work as well but have not been tested). 
To install GCC 4.6.3 type:

\texttt{sudo port install gcc46}

You will see some verbose output telling you what is happening (downloading packages, 
expanding them, building, installing, checking, etc.). When the install is complete, you can type: 

\texttt{port installed} 

to see what packages you currently have installed. You should see 
something like \texttt{gcc46 \@4.6.3\_3 (active)}. 
(The minor version numbers after the ``4.6'' may differ as MacPorts makes updates to the port.) 
You will likely also see other packages that have been installed (software dependencies for GCC 
that were automatically installed by MacPorts and/or other ports you have manually installed). 

The ``(active)" description identifies which version of a particular package Macports 
currently thinks you want to use (e.g., you could also have another older GCC suite installed). 
To make sure the newly installed version is active, you would type:

\texttt{port select gcc}

which will return something like:

\begin{verbatim}
Available versions for gcc:
   gcc40
   gcc42
   mp-gcc46 (active)
   none
\end{verbatim}

This confirms that GCC 4.6 is active (the \texttt{mp} indicates a Macports version). 
It is possible that gcc46 will be listed as active when you type \texttt{port installed}, 
but that mp-gcc46 will not be listed as active when you type \texttt{port select gcc}. 
If mp-gcc46 is not active as shown above, then you will need to select it using:

\begin{verbatim}
	sudo port select --set gcc mp-gcc46
\end{verbatim}

This will ensure that any generic call to gcc, gfortran, g++, will point to the libraries just installed.

\end{mdframed}              % ^==============MAC================^


\begin{mdframed}[style=ubuntu] % V==============UBUNTU==========V
GNU compilers may have come with your Linux distribution.  If not, they need to 
installed.  Ubuntu 12.10 comes with \texttt{gcc} installed but not \texttt{gfortran}.

Install \texttt{gfortran} with:

\texttt{sudo apt-get install gfortran}

\end{mdframed}                 % ^==============UBUNTU==========^


% =====================================
\subsection{Install build tools}
% =====================================

Additional tools are needed for managing the build process.  \texttt{make} (specifically, GNU's gmake)
usually comes with Mac and Linux distributions, but if not it should be installed.  
Additionally, CISM uses the \href{http://www.cmake.org/}{\texttt{CMake} build utility}
(a cross-platform, open-source build system).

\begin{mdframed}[style=mac] % V==============MAC================V

While you probably already have a version of \texttt{make} on your system, it may be out of date or conflict with other Macports installed software. The required versions for CISM can be installed through Macports with this command: 

\begin{verbatim}
sudo port install gmake cmake
\end{verbatim}

In addition to the software installed above, you should now see something like the following when you type \texttt{port installed}:

\begin{verbatim}
  gmake @3.82_0 (active)
  cmake @2.8.10_1 (active)
\end{verbatim}
\end{mdframed}              % ^==============MAC================^



\begin{mdframed}[style=ubuntu] % V==============UBUNTU==========V

On Ubuntu (and other Debian systems) there is usually a package called \texttt{build-essential} 
that includes a large collection of tools and libraries that are typically necessary
for compiling code. Install these tools and \texttt{CMake} with:

\texttt{sudo apt-get install build-essential cmake}
\end{mdframed}                 % ^==============UBUNTU==========^


% =====================================
\subsection{Install netCDF}
% =====================================
\label{sec:install-netcdf}
NetCDF stands for ``network Common Data Form'' libraries, which are a
 machine-independent format for representing scientific data.
This is required by CISM for performing input/output.  The netCDF package you 
install must include Fortran libraries for CISM to compile (in some package managers,
the Fortran libraries are in a separate package).  There are substantial differences 
between versions 3.x and 4.x of netCDF, but both version series should work with CISM. 
It is also possible to download and compile netCDF libraries manually, 
which may be preferred by advanced users wanting to use a specific version.

It is also recommended that you install optional tools for working with netCDF datafiles.
\href{http://meteora.ucsd.edu/~pierce/ncview_home_page.html}{\texttt{ncview}} is a
convenient tools for viewing netCDF files.  
(Some alternatives are to write Python or Matlab scripts or to use another tool like Paraview or
\href{http://ferret.pmel.noaa.gov/Ferret/home}{\texttt{Ferret}}.)  
\href{http://nco.sourceforge.net/}{\texttt{NCO}} (``netCDF Operators'') is a toolkit of command line tools
for manipulating and analyzing data stored in netCDF-accessible formats.

\begin{mdframed}[style=mac] % V==============MAC================V
To install NetCDF, use \texttt{sudo port install netcdf-fortran +gcc46}. 
Note that there are other versions of NetCDF available to install. It is important 
to choose the one with the ``Fortran" extension. The ``gcc46" syntax specifies a port ``variant". 
This tell Macports that, if there is a version of the selected software 
to install that is consistent with the GCC 4.6 compiler suite, then it should 
choose that one. Typing \texttt{port installed} should now include:

\begin{verbatim}
netcdf @4.3.2_0+dap+gcc46+netcdf4 (active)
netcdf-fortran @4.2_12+gcc46 (active)
\end{verbatim}

The ``dap+gcc46+netcdf4" comes along automatically. 

\textbf{Optional but recommended:} Tools for working with netCDF data files.

\texttt{sudo port install ncview nco}

If you encounter an `unable to open display' error when running \texttt{ncview},
you may need to install a newer version of the X Window System than the one provided by Apple.
We have had success using the latest version of XQuartz:
\href{http://xquartz.macosforge.org}{http://xquartz.macosforge.org}

\end{mdframed}              % ^==============MAC================^


\begin{mdframed}[style=ubuntu] % V==============UBUNTU==========V
Install netCDF libraries with:

\texttt{sudo apt-get install libnetcdf-dev}

\textbf{Optional but recommended:} Tools for working with netCDF data files.

\texttt{sudo apt-get install netcdf-bin ncview nco}
\end{mdframed}                 % ^==============UBUNTU==========^



% =====================================
\subsection{Install Python and related modules}
% =====================================
Python is used by CISM to autogenerate I/O code during compilation, and is also
used by most test case scripts to set up initial conditions and analyze and plot
results.  Only Python 2.7 has been tested.  Python 3 may work for some uses but is
likely to generate errors due to extensive changes between versions 2 and 3.
Also, CISM uses a number of python modules:
\begin{itemize}
  \item \texttt{numpy} - required for generating many test case initial conditions
  \item \texttt{matplotlib} - used by some plotting scripts.  Not strictly necessary but required for those scripts to work properly.
  \item  a python netCDF I/O module.  Options are \texttt{netCDF4},  \texttt{Scientific.IO.NetCDF}, or \texttt{PyCDF}.  
\texttt{netCDF4} is the ideal choice, but it is often not available through Linux package managers and must be installed through a python package manager like pip, or manually.
\texttt{PyCDF} is the least recommended option here because it is not entirely compatible with the others.  \texttt{Scientific.IO.NetCDF} is usually available through Linux package managers.
\end{itemize}


\begin{mdframed}[style=mac] % V==============MAC================V
While Mac OS X already comes with a working Python distribution, we will need 
additional modules that can sometimes be tricky to get working together correctly. 
We have successfully used both the 
\href{https://www.enthought.com/products/epd/}{Enthought Python distribution} 
(which is free for people associated with a university) and a version installed 
using Macports.  To obtain and install Enthought, click on the link above and follow their directions.
To install version 2.7 using Macports, along with the necessary 
additional modules, do the following:

\texttt{sudo port install python27 py27-numpy py27-matplotlib py27-scientific}

\texttt{          py27-netcdf4}

The existence of two versions of python on your system can lead to confusion.
It is important that you leave the version of python that came supplied by Apple so that
your system has access to it.  However, you will want to be sure that CISM has access to
the new, more modern version of python you have installed.  In our experience,
this can be one of the most problematic parts of the installation process.  
You can use \texttt{port select}:

\texttt{ sudo port select python python27}

\noindent
You can check that Macports python is used by default by typing:

\texttt{which python}

\noindent
and you should see: \texttt{/opt/local/bin/python}.  
If you instead see \texttt{/usr/bin/python} then the default Apple python is still 
the version that is being used on the command line.  If this happens, or if you 
encounter errors with this setup, an alternative approach is to modify the 
\texttt{PATH} variable in your \texttt{.bashrc} or similar environment settings script
to make sure that \texttt{/opt/local/bin} is before \texttt{/usr/bin} in your path.
\end{mdframed}              % ^==============MAC================^



\begin{mdframed}[style=ubuntu] % V==============UBUNTU==========V
Python generally comes with most Linux distributions.  If it is not present, it must be installed.
Often, there is an additional python development package that is necessary
when working with compiled code (tpyically called \texttt{python-dev} on Ubuntu).

Install python modules with:

\texttt{sudo apt-get install python-dev python-numpy python-matplotlib} 

\texttt{          python-scientific}

\textbf{Optional:} Installing \texttt{netCDF4} python module.

\noindent
Install pip (a tool for installing and managing Python packages):

\texttt{sudo apt-get install pip}

\noindent
Install \texttt{netCDF4} using pip:

\texttt{sudo -E pip install netcdf4}
\end{mdframed}                 % ^==============UBUNTU==========^



% =====================================
% =====================================
\section{Building Serial CISM}
\label{serial-build}
% =====================================
% =====================================
At this point we are ready to build a \textit{serial} version of CISM and its linked libraries. While we ultimately want to build a 
version of the code that also runs in parallel, it is often useful to stop at this step to make sure everything is working. Then, if 
problems occur during the parallel build process (as they sometimes do), we know those problems have occurred only during 
the last step of the process.

If you have not already done so, obtain the source code following the instructions above in section \ref{sec:getcode}. Below, all 
paths starting with \texttt{./} indicate the root level of the source code directory. E.g., if you expanded your tar.gz archive into a main source code 
directory with the path \texttt{/usr/JohnDoe/CISM}, the \texttt{./} refers to the path \texttt{/usr/JohnDoe/CISM/}.  

Unlike previous versions of the code, the build system is now entirely based on CMake 
(Autotools is no longer used). 

Build scripts are provided that should work for most standard Mac and Linux setups, 
as well as some supercomputing platforms on which CISM is commonly run.
All build scripts are located in the \texttt{./builds} directory from the root level of the code.
In general, change to the subdirectory that most closely matches your system and intended
build.

If you encounter an error when using the included scripts, you may need to modify some details, 
such as the location of your NetCDF libraries or your compiler names.  Other errors you might 
encounter may indicate that some of the supporting software (above) is missing.

%\textbf{SP: For the cake build to ''just work" (i.e., w/o setting an env vars, etc.), the only thing I had to do was to sym link the generic compiler names (e.g. gcc, g++, gfortran) and python in /opt/local/bin/ to the respective binary files there w/ a specific name (e.g., gcc-mp-6.4). Assuming one is in /opt/local/bin/, an example of that would be 'sudo ln -s ./gfortran-mp-4.6 ./gfortran'. This also assumes, that ones path has been pre-prended to look in /opt/local/ first, rather then /usr/local. This should have happened automatically when installing macports (it adds a line to your .profile script to cover this), but it might be worth mentioning explicitly (at the start?) as well. Note that after the fact, I found out that a more general / easier way of doing this might be to instead do, e.g., 'sudo port select --set gcc gcc46'. I have not tested this, but it is recommended (and seemed to work) for using of openmpi, e.g. by way of slaving the generic "mpi" to "openmpi-devel-gcc46-fortran". This is a faster way to do it, as it sets all the necessary sym links with a single command. I've already changed the gcc documentation above to reflect this. It would be good for someone else to confirm this if / when they do a clean install on a new machine.} 

\begin{mdframed}[style=mac] % V==============MAC================V
On a Mac, you should be able to build the code by doing the following:

\begin{enumerate}
\item{Change to the \texttt{./builds/mac-gnu-serial} directory from the root level of the code.}
\item{Configure the build using \texttt{source mac-gnu-cmake-serial}}
\item{Build the code using \texttt{make -j X}, where the ``X'' refers to the number of processors available for use in the build (or just \texttt{make} if you have only one processor).}
\end{enumerate}
\end{mdframed}              % ^==============MAC================^


\begin{mdframed}[style=ubuntu] % V==============UBUNTU==========V
On a Linux platform, you should be able to build the code by doing the following:

\begin{enumerate}
\item{Change to the \texttt{./builds/linux-gnu-cism} directory from the root level of the code.}
\item{Configure the build using \texttt{source linux-gnu-cism-cmake-serial}}
\item{Build the code using \texttt{make -j X}, where the ``X'' refers to the number of processors available for use in the build (or just \texttt{make} if you have only one processor).}
\end{enumerate}
\end{mdframed}                 % ^==============UBUNTU==========^

When the build completes, you can check for the executable driver by typing \texttt{ls cism\_driver} from within the current build directory (here, from within the \texttt{./builds/mac-gnu-serial} or \texttt{./builds/linux-gnu-cism/} directory). The file \texttt{cism\_driver} is the executable you will link to when running the model, which is generally done using a symbolic link. For example, from the \texttt{./tests/higher-order/shelf/} directory, one would link to this executable using, 

\begin{verbatim}
ln -s ../../../builds/mac-gnu/cism_driver/cism_driver ./
\end{verbatim}

Chapter \ref{ch:tests} discusses running the executable for standard test cases.

Advanced users may want more control over the build scripts.  There are a number of
build options used by CMake to customize the build.  You can manually modify the 
build scripts included with the code, or use the tool \texttt{ccmake} to 
interactively adjust build options (type \texttt{ccmake ../../} from any build directory
after having run the configure script once).  The available options are listed in Table \ref{cmake-options}.
Many of these options pertain to the parallel build which is discussed in more detail below.


\begin{table}
\begin{tabular}{ l | p{8cm} }
\hline
\texttt{CISM\_BUILD\_CISM\_DRIVER} & Toggle to build cism\_driver, on by default \\

\texttt{CISM\_BUILD\_EXTRA\_EXECUTABLES} & Toggle to build other executables, off by default \\

%\texttt{CISM\_BUILD\_SIMPLE\_GLIDE} &  Toggle to build simple\_glide, on by default \\

\texttt{CISM\_COUPLED} & Toggle to build CISM for use with CESM, off by default \\

\texttt{CISM\_ENABLE\_BISICLES} & Toggle to build a BISICLES-capable cism\_driver, off by default  \\

\texttt{CISM\_FORCE\_FORTRAN\_LINKER} & Toggle to force using a Fortran linker for building executables, off by default \\

\texttt{CISM\_GNU} & Toggle to set compilation flags needed for the gnu compiler, off by default \\

\texttt{CISM\_INCLUDE\_IMPLICIT\_LINK\_LIB} & Toggle to explicitly include the CMAKE\_Fortran\_IMPLICIT\_LINK\_LIBRARIES on the link line, on by default \\

\texttt{CISM\_MPI\_MODE} & Toggle to configure with MPI, on by default \\

\texttt{CISM\_NETCDF\_LIBS} &  netCDF library name(s) \\

\texttt{CISM\_NO\_EXECUTABLE} & Set to  ON  to just build libraries, off by default \\

\texttt{CISM\_SERIAL\_MODE} & Toggle to configure in serial mode: off by default \\

\texttt{CISM\_SOURCEMOD\_DIR} &  Path to SourceMod directory of F90 files to replace CISM files \\

\texttt{CISM\_STATIC\_LINKING} &  Toggle to set static linking for executables, off by default \\

\texttt{CISM\_USE\_DEFAULT\_IO} &  Toggle to use default i/o files rather than running python script, off by default \\

\texttt{CISM\_USE\_GPTL\_INSTRUMENTATION} & Toggle to use GPTL instrumentation, on by default \\

\texttt{CISM\_USE\_MPI\_WITH\_SLAP} & Toggle to use mpi when using SLAP solver, only relevant if CISM\_SERIAL\_MODE=ON, off by default \\

\texttt{CISM\_USE\_TRILINOS} & Toggle to use Trilinos external solver libraries, on by default \\

\texttt{CMAKE\_VERBOSE\_CONFIGURE} & Verbose CMake configuration, on by default \\

%% older options below here %%

%\texttt{CISM\_USE\_CISM\_FRONT\_END} &  Toggle to use cism\_driver or cism\_cesm\_interface with cism\_front\_end, off by default \\

%\texttt{CISM\_BUILD\_GLINT} & Toggle to build glint, off by default \\

%\texttt{CISM\_BUILD\_GLINT\_EXAMPLE} & Toggle to build glint\_example, off by default \\

%% some probably not important options here %%

%\texttt{CISM\_NETCDFF\_FOUND} & Path to NetCDF libraries  \textbf{** do we need to include this here? **} \\                                                                                                                        

\hline
\end{tabular}
  \caption{Available \texttt{CMake} settings for configuring the CISM build process}
  \label{cmake-options}
\end{table}

Also, there are standard CMake options that can be set (e.g., \texttt{CMAKE\_C\_COMPILER}, \texttt{CMAKE\_Fortran\_COMPILER}, etc.).  Many of these are explained in the \href{http://www.cmake.org/Wiki/CMake_Useful_Variables}{CMake documentation}.



% =====================================
% =====================================
\section{Installing Supporting Software for Parallel CISM}
% =====================================
% =====================================
To build parallel CISM, MPI compilers and libraries are required.  
Only the higher-order dycore (Glissade) can run in parallel.  (There is also a
higher-order dycore called Glam that can be run in parallel, but it is used
for development and testing and is not supported for scientific applications.)

In addition, you may choose to include the Trilinos package of external solver libraries.  
These are not required, but for some problems Trilinos may provide better
performance and stability than the native solvers.  Trilinos can also technically be 
used with a serial build, but this configuration is not supported or recommended.

% =====================================
\subsection{Install MPI}
% =====================================
MPI (Message Passing Interface) libraries and compilers are necessary for compiling parallel CISM.  
These libraries are used for handling parallel communications when running the 
code on multiple processors. A more complete description of parallel 
model configurations is given in Chapter \ref{ch:runcism}. 
(For example, some test cases and configurations when running the shallow-ice 
dycore are not fully supported in parallel). 
OpenMPI and MPICH are two common MPI implementations.

\begin{mdframed}[style=mac] % V==============MAC================V
It is likely that you already have versions of MPI installed on your system, 
but they may be out of date or not compatible with the other libraries we have 
and will be installing. Using Macports, the MPICH version of MPI is known 
to work when building CISM.  (OpenMPI may also work, but we've seen more 
consistent success on Macs with MPICH.)

First, check Macports for available versions of MPICH using \texttt{port search mpich*}. We want 
the version that is compatible with our GCC compiler suite, so we type: 

\begin{verbatim}
	sudo port install mpich-devel-gcc46 +fortran
\end{verbatim}

\noindent
To make sure this is active, type 
\begin{verbatim}
	port installed mpi*
\end{verbatim}

\noindent
which should return

\begin{verbatim}
     mpich-devel-gcc46 @3.2a1_0+fortran (active)
\end{verbatim}

\noindent
As when installing the GCC compilers, we want to make sure any generic call to MPI points to MPICH. This 
can be done with the following command:

\begin{verbatim}
	sudo port select --set mpi mpich-devel-gcc46-fortran
\end{verbatim}

\end{mdframed}              % ^==============MAC================^



\begin{mdframed}[style=ubuntu] % V==============UBUNTU==========V
Either OpenMPI or MPICH are likely to work with CISM on Linux machines.
On Linux machines, we have tested OpenMPI more thoroughly.
Install OpenMPI with:

\texttt{sudo apt-get install openmpi-bin}
\end{mdframed}                 % ^==============UBUNTU==========^



% =====================================
\subsection{Install Trilinos solver libraries}
\label{sc:install_trilinos}
% =====================================

Trilinos is a modern, open source, C++ based library of parallel nonlinear and linear solvers, 
preconditioning and mesh-partitioning tools, and much more. It can be downloaded 
\href{trilinos.sandia.gov/download/}{here}.
(The software is free, but you are required to enter your email address to download it.) 
The documentation below assumes that you are working with version 11.10.* and was specifically 
tested using version 11.10.2. 

Building Trilinos requires CMake version 2.8 or later, which ideally you have already 
installed as discussed above. Trilinos is not needed to run the default 
parallel, higher-order dycore (Glissade), but it may be useful for  
more difficult problems or for debugging in cases where the native Fortran solvers 
fail to converge.

The build instructions for Trilinos on Mac and Linux are very similar, so users
of both systems can follow the primary instructions below, except where noted.

Trilinos requires both (1) an ``out-of-source build'' and (2) an ``out-of-build installation''. 
This means that you cannot build the code in the same directory where the source code lives, 
and you cannot install the libraries in the same directory where you build the code.
(Older versions of Trilinos required an out-of-source build but not an out-of-build installation.) 
The easiest way to satisfy this requirement is to have separate ``source'', ``build'' and 
``install'' directories in the location where you want to install the code. 
For example, in \texttt{/usr/local/}, you could set up the following three directories:

\begin{verbatim}
	trilinos-11.10.2-Source/
	trilinos-11.10.2-Build/
	trilinos-11.10.2-Install/
\end{verbatim}

The ``source'' directory will be created on its own when you uncompress the tar.gz archive 
that you download. You do not have to keep the source code where you build and install 
the Trilinos libraries, but you will need to remember the path to where that source code 
lives on your computer. 

To configure the Trilinos build, you will need to execute a CMake configure script. 
Sample configure scripts for a number of standard platforms are included in the ``sampleScripts''
directory under the root level of the Trilinos source code. 
Also, the CISM code includes examples of Trilinos configure scripts (``do-configure'') 
for use with CISM for both Linux and Mac platforms in the 
\texttt{./utils/trilinos\_config\_scripts\_examples} directory. 
We recommend starting with one of those scripts and modifying it as
necessary to work on your system\footnote{If you are following the above installation instructions for
Mac exactly, then the configure script 
\texttt{./utils/trilinos\_config\_scripts\_examples/do-configure-Trilinos-11.10.2-for-Mac-10.9.4} should 
work with few modifications.}.

The paths to both the ``source'' and ``install'' directories are specified within the ``do-configure'' scripts. In these instructions, those directories are both assumed to live within \texttt{/usr/local/},
but other locations are fine to use too (e.g., in your home/User directory).

\begin{mdframed}[style=mac] % V==============MAC================V
Also note the explicit path in the MPI lines, e.g.,

\begin{verbatim}
-D MPI_EXEC="/opt/local/mpiexec" \
\end{verbatim}

Since some Macs may come with their own pre-installed OpenMPI libraries, it is important here to specify the path to the version we previously installed using Macports.
\end{mdframed}              % ^==============MAC================^

Find the example script most appropriate for your system, copy it to the \texttt{trilinos-11.10.2-Build} directory, and modify it if necessary (e.g., adjust paths, compiler locations, etc.).
Execute it with: 

\texttt{source ./do-cmake}\footnote{Here we have assumed that the name of the configure script is ``do-cmake''. The script name may differ depending on what you have called it or if you copied and modified one of the scripts from \texttt{./utils/trilinos\_config\_scripts\_examples}.}

\noindent
from within your \texttt{trilinos-11.10.2-Build} directory. Depending on where you are building and installing the code, you may need to have administrative privileges (in which case you would type \texttt{sudo source ./do-cmake}). If the configure step was successful, you should see the following displayed on your screen:

\begin{verbatim}
...
Processing enabled package: [PACKAGE NAME]
...

Exporting library dependencies ...

Finished configuring Trilinos!

-- Configuring done
-- Generating done
-- Build files have been written to: /usr/local/trilinos-11.10.2-Build
\end{verbatim}

It is a good idea to scan the output while the ``do-cmake'' script is executing, 
for example to ensure the configure process is picking up the compilers you specified 
(e.g., it is using the Macports versions as opposed to some Mac default versions that 
might also be on your system). Once the code is configured successfully, build the libraries 
from within the \texttt{trilinos-11.10.2-Build} directory by typing:

\texttt{make} (or \texttt{sudo make} if necessary) 

\noindent
For multiprocessor machines, the build process can be sped up significantly using 
the ``-j''command as described above for building serial CISM:

\texttt{make -j X}

\noindent
where ``X'' is the number of cores available on your machine (e.g., \texttt{make -j 4} 
for a 2-processor, dual-core machine).

Building Trilinos can take a long time (e.g., an hour or more), depending on your machine, 
the number of processors used for the build, and the number and type of libraries 
you are installing. Once you have built the code, we highly recommend testing it 
using:

 \texttt{make test} 

\noindent
(The \texttt{Trilinos\_ENABLE\_TESTS:BOOL}
variable in the do-cmake script can be set to ``OFF''to disable building of the tests.)
Screen output will tell you if and how many tests failed. We have seen 
a few tests fail while still having a perfectly good and working Trilinos library. 
In general, if the number of tests passed is above 90\%, the library will likely work fine with CISM.
Query the CISM \href{mailto:cism-users+subscribe@googlegroups.com}{users} or
\href{mailto:cism-devel+subscribe@googlegroups.com}{developers} lists\footnote{cism-users+subscribe@googlegroups.com -or- cism-devel+subscribe@googlegroups.com} 
if you have questions about specific Trilinos tests failing.

\begin{mdframed}[style=mac] % V==============MAC================V
On a Mac, MPI tests have been known to trigger a dialog box from the firewall. 
With more than 300 tests, these messages popping up continually can make it impossible 
to use your computer until the tests complete. To keep them from appearing, you can temporarily 
turn off your firewall under ``System Preferences'' (Security $>$ Firewall $>$ Stop). 
Be sure to turn the firewall back on when the tests are complete!
\end{mdframed}              % ^==============MAC================^

After running the tests, you will need to install Trilinos using:

 \texttt{make install}

\noindent
This will build the actual Trilinos libraries in the path specified in the

\begin{verbatim}
-D CMAKE_INSTALL_PREFIX:PATH=/path
\end{verbatim} 

\noindent
line of your ``do-cmake'' script (above). For this example, those libraries will be 
installed in: \texttt{/usr/local/trilinos-11.10.2-Install}

After successfully building Trilinos, create an environment variable called \texttt{CISM\_TRILINOS\_DIR}
so the CISM build process can find the Trilinos installation.  For example, if you 
are using the bash shell and your current directory is the Trilinos install directory, you can do:
\begin{verbatim}
export CISM_TRILINOS_DIR=$PWD
\end{verbatim}
You may prefer to modify your .bashrc or .bash\_profile (or similar)
to set this environment variable on every login.

Alternatively, you can modify the CISM parallel build script (below) so that the line:
\begin{verbatim}
-D CISM_TRILINOS_DIR=$CISM_TRILINOS_DIR \
\end{verbatim}
is set to the Trilinos installation directory.





% =====================================
% =====================================
\section{Building Parallel CISM}
% =====================================
% =====================================

The procedure for building parallel CISM is nearly identical to the serial build (above).
The build script for parallel CISM for a Mac is located at \texttt{builds/mac-gnu/mac-gnu-cmake}, 
while the build script for Linux is located at \texttt{builds/linux-gnu-cism/linux-gnu-cism-cmake}.
From the appropriate directory, run:

\texttt{ source mac-gnu-cmake} or \texttt{ source linux-gnu-cism-cmake}

\noindent{Once the configuration step completes successfully, you can compile the code as before with:}

\texttt{make}

\noindent{or}

\texttt{make -j 4}

\noindent{if you have 4 processors available (or as many processors as you would like to use).
See Section \ref{serial-build} for details about customizing the build process.

Building a parallel version of CISM that includes Trilinos requires setting the  
\begin{verbatim}
-D CISM_USE_TRILINOS 
\end{verbatim}
\noindent{flag to \texttt{ON} in the \texttt{builds/mac-gnu/mac-gnu-cmake} script.}

% =====================================
% =====================================
\section{Next Steps}
% =====================================
% =====================================
If you make any changes to the source code, you only need to re-run
\texttt{make} from your build directory to generate an updated executable.  
One exception is that if you edit the lists of input/output netCDF variables in 
\texttt{*\_vars.def} files (see Appendices \ref{ug.sec.varlist} and \ref{ch:appendix_io}), 
you need to first re-source the configuration script 
(e.g., \texttt{source mac-gnu-cmake}) before re-running \texttt{make}.


Now that you have successfully built the code, you can proceed to Chapters \ref{ch:modelingintro} and \ref{ch:higher-order} to learn more detailed 
information about ice sheet modeling, to Chapters \ref{ch:glide} and \ref{ch:glissade} to learn more about the various model approximations 
available through CISM, or you can proceed to Chapter \ref{ch:tests} to learn how to run and examine some standard model test cases.  


